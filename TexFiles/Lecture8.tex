\begin{definition}
Let $U$ and $Y$ Banach spaces and $J: U\rightarrow \overline{\mathbb{R}}$ is a proper function. We consider an optimization problem \eqref{eq8. P} called primal problem
 \[
 \inf_{u\in U} J(u) \tag{P} \label{eq8. P}
 \] 
 Then the problem is said to be nontrivial if there is $\overline{u} \in U$ such that $J(\overline{u}) < \infty$. A function $\Phi : U \times Y \rightarrow \overline{\mathbb{R}}$ is said to be a perturbation function of $J$, 
 \[
 \inf_{u\in U} \Phi (u,p) \tag{Pp}\label{eq8. Pp}
 \]
 if $\Phi(u, 0)=J(u)$ for all $u \in U$. For each $p \in Y$, the minimization problem \eqref{eq8. Pp} is called a perturbation problem. The variable $p$ is called perturbation parameter. If we denote by $\asterisk{\Phi}$ the convex conjugate function of $\Phi$, the \textit{dual problem}, with respect to $\Phi$ is defined by
 \[
 \sup_{\asterisk{p}\in \asterisk{Y}}-\asterisk{\Phi}(0,\asterisk{p}) \label{eq8. P*}\tag{P*}
 \]
 
 where $\asterisk{\Phi}: \asterisk{(U\times Y)} \cong \asterisk{U}\times\asterisk{Y}\rightarrow \overline{\mathbb{R}}$, a function defined as follows.
 \[
 \asterisk{\Phi}\left(\asterisk{u}, \asterisk{p} \right) = \sup_{\substack{u \in U\\ p\in Y}} \left\lbrace \asterisk{u}(u)+\asterisk{p}(p)-\Phi(u,p)\right\rbrace
 \] 

\end{definition}
 \begin{remark}
For $p=0$, \eqref{eq8. P*} $\equiv$ \eqref{eq8. Pp}. We denote the infimum for problem \eqref{eq8. P} by $\inf\eqref{eq8. P}$ and the supremum for problem \eqref{eq8. P*} by $\sup \eqref{eq8. P*}$
 \end{remark}

 
 \begin{lemma}[Weak duality]
 	For the problem \eqref{eq8. P} and \eqref{eq8. P*} it holds that
 	\[
 	 -\infty \leq \sup\eqref{eq8. P*} \leq \inf\eqref{eq8. P} \leq \infty.
 	\]
 	\begin{proof}
 		Let $\asterisk{p}\in\asterisk{Y}$. It follows
 		\begin{align}
 			-\asterisk{\Phi}(0,\asterisk{p})=&-\sup_{\substack{u\in U\\ p\in Y}}\left\lbrace 0(u) + \asterisk{p}(p)-\Phi(u,p)
 			\right\rbrace\\
 			=&\inf_{\substack{u\in U\\p\in Y}} \left\lbrace\Phi(u,p)-\asterisk{p}(p)\right\rbrace\\
 			\leq & \Phi(u,0)-\asterisk{p}(0)\quad \forall u \in U, \asterisk{p} \in \asterisk{Y}\\
 			\implies \sup_{ \asterisk{p}\in \asterisk{Y}}\left\lbrace-\Phi\left(0, \asterisk{p}\right) \right\rbrace \leq&\inf_{u\in U} \Phi (u, 0) = \inf(P)
 		\end{align}
 	 \end{proof}
 	\end{lemma}
 	By iteration we can define, a bidual problem 
 	\[
 	-\sup_{u \in U} \left\lbrace-\asterisk{\Phi}(u,0)\right\rbrace=\inf_{u\in U} \asterisk{\Phi}(u,0) \tag{P**} \label{eq8. BiDual Problem}
 	\]
 	In case the space $U$ is reflexive then $\asterisk{\asterisk{U}}=U$.
 	
 	If the perturbation function $\Phi(u,p)$ is proper, convex and weakly lower semicontinuous. Then $\asterisk{\asterisk{\Phi}}=\Phi$. In this case $\Phi (u, 0) =\asterisk{\asterisk{\Phi}}(u, 0)$ i.e \eqref{eq8. P}$\equiv$ \eqref{eq8. BiDual Problem}

  \begin{definition}
  		Consider the infimal value function
  		\[
  		h(p)=\inf \eqref{eq8. Pp} = \inf_{u \in U} \Phi(u,p)
  		\]
  	The problem \eqref{eq8. P} is called stable if $h(0)$ is finite and its sub-differentiable in zero is not empty.
  	\end{definition}
  	
  \begin{theorem}
  	The primal problem \eqref{eq8. P} is stable 
  	if and only if the following conditions are simultaneously satisfied:
  	\begin{itemize}
  		\item The dual problem \eqref{eq8. P*} has a solution.
  		\item There is no duality gap, i.e.
  		\[ 
	  		\inf\eqref{eq8. P}= \sup\eqref{eq8. P*} \leq\infty
  		\]
  	\end{itemize}
 \end{theorem}	

\begin{theorem}[Extremal relation]
	Let $\Phi:U\times Y \rightarrow \overline{\mathbb{R}}$, be convex the the following statements are equivalent:
	
	\begin{enumerate}[]
		\item \eqref{eq8. P} and \eqref{eq8. Pp} have solutions $\overline{u}$ and $\overline{\asterisk{p}}$ and $\inf(P)=\sup(\asterisk{P})$
		\item $\Phi(\overline{u}, 0)+\asterisk{\Phi}\left(0, \overline{\asterisk{p}}\right)=0$
		\item $\left(0, \overline{\asterisk{p}}\right) \in \partial \Phi (u,0)$ and $ \left(\overline{u}, 0\right) \in \partial \asterisk{\Phi}(0,\asterisk{p})$
	\end{enumerate}
	\begin{proof}
		We proceed by parts:
		\begin{enumerate}
			\item (1)$\implies$ (2): $\overline{u}$ solution of $\inf\eqref{eq8. P}$ and $\overline{\asterisk{p}}$ solution of $\sup\eqref{eq8. P*}$ and $\inf\eqref{eq8. P}=\sup\eqref{eq8. P*}$. This properties implies, $
			\Phi(\overline{u},0)=\inf\eqref{eq8. P}=\sup\eqref{eq8. P*}=-\Phi(0,\overline{\asterisk{p}})\implies \Phi(\overline{u}, 0)+\asterisk{\Phi}(0, \overline{\asterisk{p}}) =0$.
			\item (2) $\implies$ (1): $-\asterisk{\Phi}(0, \overline{\asterisk{p}})=\sup\eqref{eq8. P*}\leq \inf\eqref{eq8. P}=\Phi(\overline{u}, 0)=-\asterisk{\Phi}(0, \overline{\asterisk{p}}) \implies \sup\eqref{eq8. P*}=\inf\eqref{eq8. P}$
			\item (2)$\iff$(3): $\Phi(\overline{u},0)+\asterisk{\Phi}(0, \overline{\asterisk{p}})=0=(0, \overline{u}) +(\overline{\asterisk{p}},0)=\left((0, \overline{\asterisk{p}}), (\overline{u},0)\right) \iff (0, \overline{\asterisk{p}}) \in \partial\Phi(\overline{u},0) \ \forall u \in U,  \forall \asterisk{p} \in \partial J(u) \iff J(u)+\asterisk{J}(\asterisk{p})=(\asterisk{p}, u)$
		\end{enumerate}
	\end{proof}
\end{theorem}
\paragraph{Fencel duality.}
Consider the functional $J:U\rightarrow\overline{\mathbb{R}}$, 
\[
	J(u)=F(u)+G(Au)
\]
 with $F:U\rightarrow\overline{\mathbb{R}}$,  G convex function $G: V\rightarrow \overline{\mathbb{R}}$ and $A:U\rightarrow V$ bounded and linear.
 
 We introduce the perturbation $\Phi(u, p) = F(u)+G(Au-p)$. The dual problem is obtained with,
 \[
	 \asterisk{\Phi}(0, \asterisk{p})=\sup_{\substack{u\in U\\ p \in V} }\left\lbrace \asterisk{p}(p)-F(u)-G(Au-p)\right\rbrace
 \]
 For fixed $u$ we set $q: Au -p$.
 
 \begin{align*}
 \asterisk{\Phi}(0, \asterisk{p})=&\sup_{u\in U} \sup_{q\in V} \left\lbrace\asterisk{p}\left(Au-q\right)-F(u)-G(q)\right\rbrace\\
 =&\sup_{u\in U} \sup_{q\in V} \left\lbrace\asterisk{p}\left(Au\right)-\asterisk{p}(q)-F(u)-G(q)\right\rbrace\\
 =&\sup_{u\in U}  \left\lbrace\asterisk{p}\left( Au\right)-F(u)\right\rbrace+\sup_{q\in V}\left\lbrace(-\asterisk{p})(q)-G(q)\right\rbrace\\
  =&\sup_{u\in U}  \left\lbrace(\asterisk{A}\circ\asterisk{p})\left(u\right)-F(u)\right\rbrace+\sup_{q\in V}\left\lbrace(-\asterisk{p})(q)-G(q)\right\rbrace\\
 =&\asterisk{F}(\asterisk{A}\circ\asterisk{p})+\asterisk{G}(-\asterisk{p})
 \end{align*}
Where $(A^*\circ p^* )\in U^{*}$, defined as $(A^*\circ p^*):U\rightarrow \overline{\mathbb{R}}$ \[
	 (A^*\circ p^*)(u)= p^*(Au)
\]

In case $U$ is a Hilbert space $A^*$ is the adjoint operator of $A$.
