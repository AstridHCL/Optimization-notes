\begin{fact}
	Let $x\in H$,  let $U$ be a neighborhood of $x$, let $G$ be a real Banach space, let $T:U\rightarrow G$, let $V$ be a neighborhood of $Tx$, and let $R : V \rightarrow K$.
	Suppose that $T$ is Fréchet differentiable at $x$ and that $R$ is Gâteaux differentiable at $Tx$. Then $R\circ T$ is Gate\^aux differentiable at x and $\mathsf{D}(R \circ T )(x) =
	(\mathsf{D}R(Tx))\circ \mathsf{D}T (x)$. If R is Fr\'echet differentiable at x, then so is $R \circ T $.
\end{fact}

\begin{fact}
	Let $x \in H$, let $U$ be a neighborhood of $x$, let $K$ be a real Banach
	space, and let $T: U \rightarrow K$. Suppose that $T$ is twice Fr\'echet differentiable at x.
	Then $\forall(y, z) \in H \times H$,  $(\mathsf{D}^2 T (x)y)z = (\mathsf{D}^2 T (x)z)y$.
\end{fact}

\begin{definition}
	Let $x \in H$, let $C \in \mathcal{V}(x)$, and let $T : C \rightarrow K$. Then T is
	Fr\'echet differentiable at $x$ if there exists an operator $\mathbf{D}T (x) \in B(H, K)$, called the Fréchet derivative of $T$ at $x$, such that
	\[
	\lim_{0\neq\norm{y}\rightarrow 0}\frac{\norm{T(x+y)-Tx-\mathsf{D}T(x)y}}{\norm{y}}=0
	\]
\end{definition}