\subsection{Infinite-Dimensional Optimization} 
Let $(U,d)$ be a metric space and $J:U\rightarrow \overline{\mathbb{R}}$. We call a minimization problem.
\[
	\min_{u\in C} J(u)
\]


\begin{definition}
	A point $u\in U$ is called:
	\begin{itemize}
		\item \textbf{Local Minimizer.} If there is a neighborhood $V \in U$ such that $J(u) \leq J(v)$, $\forall v \in V$.
		\item \textbf{Global Minimizer. } If $J(u)\leq J(v)$, $\forall v \in U$.
	\end{itemize}
\end{definition}
\begin{definition}
	Let be $\{u_k\} \in U$, a convergent sequence in $U$, such that converges to $u \in U$. The functional $J$ is called lower semicontinuous at $u\in U$ if
	\[
		J(u) \leq \liminf_{k\rightarrow\infty} J(u_k).
	\]
	In general if $J$ is lower semicontinuous at $u$, for all the $u \in U$. $J$ is lower semicontinuous (l.s.c). 
\end{definition}

\begin{theorem}
	Let $J: U\rightarrow \mathbb{R}$ lower semicontinuous functional and $\exists \xi \in \mathbb{R}$, such that the level set $\mu_\xi=\{ u \in U \mid J(u)\leq \xi\}$ be non-empty and compact set of $U$. Then there exists a global minimum.
	\begin{proof}
		Let $\alpha := \inf_{u\in U} J(u)$. Then $\exists \{u_n\} \in U $ such that $J(u_n) \rightarrow \alpha$. Then  $\exists N\in \mathbb{N}$, such that $\forall k \geq N$,  $J(u_k) \leq r$ (otherwise $r=\alpha$), then we have since $\mu_\xi$ is not empty, $u_k \in \mu_\xi$. Since $\mu_\xi$ is compact,  $\exists \{u_k\}_l$ a subsequence of $\{u_k\}$ that converges in $\mu_\xi$, i.e. $\{u_k\}_l \rightarrow \overline{u} \in \mu_\xi$, as $l \rightarrow \infty$	. Since $\alpha$ is the infimimum and $J$ is lower semicontinuous and,
		 \[\alpha \leq J(\overline{u}) \leq \liminf_{l\rightarrow \infty} J(u_{k_l}) \]
		 On the other hand,  since $J(u_k) \rightarrow \alpha$,
		 \[
		  \liminf_{l\rightarrow \infty} J(u_k) \leq \alpha
		 \]
		 Therefore $J(\overline{u})=\alpha$, and hence $\overline{u}$ exists and it is a global minimizer.
	\end{proof}
	\label{th1. Existence of global minimizer}
\end{theorem}
\begin{corollary}
	Let $U$ be a Banach space. If the following conditions hold:
	\begin{itemize}
		\item $\exists \mu_\xi \in U$ (level set) non-empty and compact.
		\item $J: U \rightarrow \overline{\mathbb{R}}$ is lower semicontinuous. 
	\end{itemize}
	Then set of global minimizers  $G$ is compact.
	\begin{proof}
		The theorem \ref{th1. Existence of global minimizer} implies that all minimizers are in the set $\mu_\xi$. Therefore by proposition \ref{prop0. closed subset compact. Precompact}, $G$ is precompact. Since $J$ is lower semicontinuous, for any convergent sequence $(u_k) \in G$, we have
		\[
			\alpha \leq J(u) \leq \liminf_{k\rightarrow\infty} J(u_k)=\alpha
		\]
		Implying that the limit is also a global minimizer. Hence $G$ is closed.
	\end{proof}
\end{corollary}
\subsection{Derivatives}
Let $U$ and $V$ Banach spaces and $F:U\rightarrow V$ a mapping from $U$ to $V$ (that could be non linear).
\begin{definition}
	Let $C$ be a subset of $U$, let $F: C \rightarrow V$, and let $x \in C$
	be such that, for all $y \in U$, $\exists \alpha >0$ and the set $[x, x + \alpha y] \subset C$. Then $F$ is G\^ateaux
	differentiable at $x$ if there exists an operator $\mathsf{D}F(x) \in \mathcal{B}(U, V)$, called the G\^ateaux derivative of $F$ at $x$, such that,
	\[
	\forall (y \in U)\quad \mathsf{D}F(x) = \lim\limits_{\alpha \downarrow 0} \frac{F(x+\alpha y)-F(x)}{\alpha}
	\]
	
	 Thus, the second G\^ateaux derivative of $F$ at $x$ is the operator $\mathsf{D}^2 F(x) \in \mathcal{B}(U, \mathcal{B}(U, K))$ that
	 satisfies \[(\forall y \in  U) \quad \mathsf{D}^2 F(x) y = \lim\limits_{\alpha \downarrow 0} \frac{\mathsf{D}F(x + \alpha y) - \mathsf{D}F(x)}{\alpha} 
	 \]
\end{definition}
 
\begin{remark}
	The G\^ateaux derivative $\mathsf{D}F(x)$ is unique whenever it exists.	
\end{remark}

\begin{definition}
	Let $x \in U$, let $C$ a set contained  in a neighborhood $\mathcal{V}(x)$ de $x$, and let $F : C \rightarrow V$. Then $F$ is
	Fr\'echet differentiable at $x$ if there exists an operator $\mathsf{D}F(x)\in \mathcal{B}(U, V)$, called the Fr\'echet derivative of $F$ at $x$, such that
	\[
	\lim_{0\neq\norm{y}\rightarrow 0}\frac{\norm{F(x+y)-F(x)-\mathsf{D}F(x)y}}{\norm{y}}=0.
	\]
	
	Higher-order Fr\'chet derivatives are defined inductively. Thus, the second Fr\'echet derivative of $F$ at $x$ is the operator $\mathsf{D}^2 F(x) \in  \mathcal{B}(U, \mathcal{B}(U, V))$ that satisfies,
	\[
		\lim_{0\neq\norm{y}\rightarrow 0}\frac{\norm{\mathsf{D}F(x+y)-\mathsf{D}F(x)-\mathsf{D}^2F(x)y}}{\norm{y}}=0.
	\]
	\label{def1. Frechet.}
\end{definition}


\begin{lemma}
	Let $x \in U$, let be $C$ a set $\mathcal{V}(x)$ contained in a neighborhood of $x$, and let $F: C \rightarrow V$. Suppose that $F$ is Fr\'echet differentiable at $x$. Then the following hold:
	\begin{itemize}
	\item  $F$ is G\^ateaux differentiable at $x$ and the two derivatives coincide.
	\item $F$ is continuous at $x$.
	\end{itemize}
	\begin{proof}
		Denote the Fr\'echet derivative of $F$ at $x$ by $L_x$.
		\begin{itemize}
			\item Let $\alpha >0$ and $y \in U\backslash\{0\}$. Then
			
			\[
				\norm{\frac{F(x+\alpha y)-Fx}{\alpha}-L_xy} = \norm{y}\frac{\norm{F(x+\alpha y)-Fx-L_x(\alpha y)}}{\norm{\alpha y}}
			\]
			converges to $0$ as $\alpha \downarrow 0$, since $F$ is Fr\'echet differentiable.
			\item Fix $\epsilon > 0$. By definition \ref{def1. Frechet.}, we can find $\delta \in  (0, \frac{\epsilon}{\epsilon + \norm{L_x}}]$, such that for all $y$ in the open ball of radius $\delta$ and center in zero, (i.e. $\forall y \in B_\delta(0)$), 
			\[
			   \norm{F(x+y)-Fx-L_xy} \leq \epsilon \norm{y}\]
			Thus, $\forall y \in B_\delta(0)$,  by triangle inequality, 
			 \begin{align*}  
			\norm{F(x+y)-Fx} &\leq \norm{F(x+y) -Fx-L_xy} + 
			\norm{L_xy} \\ &\leq \epsilon \norm{y} + \norm{L_x }\norm{y} \\ &\leq \delta(\epsilon + \norm{L_x})
			\\
			&\leq \epsilon.
			\end{align*}
			It follows that $F$ is continuous at $x$.
		\end{itemize}
	\end{proof}
\end{lemma}



\begin{fact}
	Let $x\in U$,  let $\mathcal{U}$ be a neighborhood of $x$, and let $G$ be a real Banach space, let $F: \mathcal{U}\rightarrow G$ a mapping from $\mathcal{U}$ to $G$, let $\mathcal{V}$ be a neighborhood of $Tx$, and let $R : V \rightarrow K$.
	Suppose that $T$ is Fr\'echet differentiable at $x$ and that $R$ is G\^ateaux differentiable at $Tx$. Then $R\circ T$ is Gate\^aux differentiable at x and $\mathsf{D}(R \circ T )(x) =
	(\mathsf{D}R(Tx))\circ \mathsf{D}T (x)$. If R is Fr\'echet differentiable at x, then so is $R \circ T $.
\end{fact}

\begin{fact}[]
	Let $x \in U$, let $\mathcal{U}$ be a neighborhood of $x$, let $G$ be a real Banach
	space, and let $F: U \rightarrow U$. Suppose that $F$ is twice Fr\'echet differentiable at $x$.
	Then $\forall (y,z) \in U \times U$,  $(\mathsf{D}^2 F (x)y)z = (\mathsf{D}^2 F (x)z)y$.
\end{fact}

\begin{remark}[Notation]
	The result of applying the operator $\mathsf{D}F(x)$ (derivative at $x$) applied to some vector $y$ will be denote by apostrophe to the function $F'(x; y)=\mathsf{D}F(x)y$. If we are in a Hilbert Space $(H, \langle \cdot \rangle)$ by Riesz representation formula and the fact $\mathsf{D}F(x) \in \mathcal{B}(H,H)$, we have
	\[
		F'(x; y)=\mathsf{D}F(x)y=\langle y, \nabla F(x) \rangle,
	\]
where $\nabla F(x) \in H$ and it represents the operator $\mathsf{D}F(x)$. Sometimes we also denote $F'(x)=\nabla F(x)$.
\end{remark}

