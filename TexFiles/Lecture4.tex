\begin{lemma}
	Let U be linear space and $J: U\rightarrow \overline{\mathbb{R}}$. Then
	\begin{enumerate}
		\item If $J$ is convex, then the effective domain $\mathrm{dom}(J)=\lbrace u\in U | \ J(u) < \infty\rbrace $ is convex.
		\item $J$ is convex $\iff$ $\mathrm{epi}(J)=\left\lbrace (u,\alpha) \in U\times \mathbb{R} | J(u)\leq \alpha\right\rbrace$ is convex.
	\end{enumerate}
	\begin{proof} Since $U$ and $\mathbb{R}$ are linear spaces, is easy to see that scalar multiplications and sums are well defined over $U\times \mathbb{R}$ and so over $\epi(J)$.
		\begin{enumerate}
			\item Assume $J$ convex. If $u_1\in \dom (J)$ and $u_2$ are elements of $\dom (J)$. Therefore, $J(u_1)<\infty$, and $J(u_2)<\infty$, therefore for $t \in [0,1]$, we have $tJ(u_1)<\infty$ and $(1-t)J(u_2)<\infty$. Since $J$ is convex, 
			\[J(t u_1 + (1-t)u_2)\leq tJ(u_1)	+(1-t)J(u_2) < \infty\],
			
			Therefore, $tu_1 +(1-t)u_2 \in \dom (J)$. Hence $\dom J$ is convex.
			\item First consider $J$ a convex functional, then we have for all $u_1, u_2 \in U$,
				\[
					J(t u_1 + (1-t)u_2)\leq tJ(u_1)	+(1-t)J(u_2) \quad \forall t \in [0,1]
				\]
				Let $(u_1, \alpha_1)$, $(u_2, \alpha_2)$ elements of $\epi(J)$, then $J(u_1) < \alpha_1$ and $J(u_2)< \alpha_2$. Since $J$ is convex. 
				\[
					J(t u_1 + (1-t)u_2) \leq t \alpha_1 + (1-t)\alpha_2
				\]
				Then $(tu_1+(1-t)u_2, t \alpha_1+(1-t)\alpha_2) \in \epi(J)$. Therefore, if $J$ is convex, and $(u_1, \alpha_1)$, $(u_2, \alpha_2)$ are elements of $\epi(J)$ then,
				\[ t(u_1,\alpha_1)+(1-t)(u_2, \alpha_2)=(tu_1+(1-t)u_2, t \alpha_1+(1-t)\alpha_2) \in \epi(J) \]
				Hence $\epi(J)$ is convex. 
				
				Now assume $\epi(J)$ convex. Let $(u_1, \alpha_1)$, $(u_2, \alpha_2)$ elements of $\epi(J)$ then $(tu_1+(1-t)t \alpha_1 + (1-t)\alpha_2) $, then
				\[
							J(t u_1 + (1-t)u_2) \leq t \alpha_1 + (1-t)\alpha_2 \quad \forall t \in [0,1]
				\]
				
				By definition of $\epi(J)$, if $u_1, u_2 \in \dom J$, then $(u_1, J(u_1))$ and $(u_2, J(u_2))$, are elements of $\epi(J)$, therefore
				\[
					J(t u_1 + (1-t)u_2) \leq t J(u_1) + (1-t)J(u_2) \quad \forall t \in [0,1]
				\]
				Implying that $J$ is convex.
		\end{enumerate}
	\end{proof}
\end{lemma}

\begin{definition}
	Let $U$ a Banach space. Then the function $J:U\rightarrow \overline{\mathbb{R}}$ is called lower semi-continuous at $u_0 \in U$ if the following conditions holds:
	\begin{itemize}
		\item If $\forall \epsilon > 0$ there is a neighborhood $B_\delta (u_0)$ of $u_0$ such that $J(u_0)-\epsilon \leq J(u) \ \forall u \in B_\delta{u_0)}$.
		\item If $J(u_0)\leq\liminf_{n\rightarrow \infty} J(u_n)$ holds for each sequence $u_n \in U$.
	\end{itemize}
\end{definition}

\begin{remark}
	If the second condition holds, $J$ is called sometimes sequentially semi-continuous. If $J$ is continuous it is also lower semi-continuous.
\end{remark}
\begin{theorem}
	\label{th4: levelsets and lsc}
	Let $U$ be a Banach space and $J: U\rightarrow \overline{\mathbb{R}}$. Then sthe following conditions are equivalent.
	\begin{enumerate}
		\item $J$ is lower semi-continuous, i.e., $J$ is lower semi-continuous at every point in $U$.
		\item The $\epi(J)$ is closed.
		\item The level sets $\mu_\xi=\lbrace u \in U | J(u)\leq \xi \rbrace$ is a closed set. Note that  the sets $\mu_\xi$ are closed if and only if the sets $\gamma_\xi=\lbrace u \in U | J(u)> \xi \rbrace $ are open. (Since $\mu_\xi ^c = \gamma _\xi$).

	\end{enumerate}
	\begin{proof}\
		
		\begin{itemize}
			\item $(1)\implies(2)$ Let $(u_n, \xi_n)$, be a sequence in $\epi(J)$, such that converges to $(u,\xi)$ in $U\times\mathbb{R}$. Then \[J(u)\leq \liminf_{n\rightarrow \infty} J(u_n) \leq \liminf_{n\rightarrow \infty}\xi_n=\xi. \]Hence $(u, \xi) \in \epi (J)$.
			
			\item $(2)\implies (3)$Let $\xi\in \mathbb{R}$ and assume that $(u_n)_{n\in \mathbb{N}}$ is a sequence in $\mu_\xi$ that converges to $u$. 
			Then the set $(u_n,\xi)_{n\in\mathbb{N}}$ is in $\epi (J)$. Since $\epi (J)$ is closed, we conclude that $(u, \xi)\in \epi (J)$, and hence $u \in \mu_k$.
			
			\item $(3)\implies (1)$ Let bet $u \in U$ an arbitrary member of the Banach space $U$, and let $(u_n)_{n\in \mathbb{N}}$ be a sequence that converges to $u$. And we set the number $\eta=\liminf_{n\rightarrow \infty} J(u_n)$. Then we have to prove that $J(u)\leq \eta$. When $\eta=\infty$, the inequality is clear. Therefore we assume that $\eta < +\infty$. Since every sequence in $\mathbb{R}$ has a subsequence that converges to the $\liminf$, the sequence $(u_n)_n$ has a subsequence $(u_k)_k$, such that $J(u_k) \xrightarrow{k\rightarrow\infty} \eta$. Now, we can fix $\xi \in (\eta, \infty)$. By convergence we can find $c$ such that $k\geq c$ implies that $(J(u_k))$ belongs to $(-\infty, \xi)$, therefore the set \[\lbrace u_k | \ k \geq c \in \mathbb{N}\rbrace \subset \mu_\xi. \]
			
			Since the sequence $u_n \rightarrow u$, the subsequence $u_k\rightarrow u$. And $\mu_\xi$ closed implies $u \in \mu_\xi$. Since this holds for all $\eta < \infty$, we take $\xi \downarrow \eta$. Implying $J(u)\leq \eta$. 
		\end{itemize}
	\end{proof}
\end{theorem}
\begin{example}
The indicator function of a set $C \subset U$, i.e. the function $I_C: U\rightarrow[-\infty, \infty]$
\begin{equation*}
I_C(u)=
\left\lbrace
	\begin{array}{rl}
	0, & \text{if } u \in C \\
	+\infty, & otherwise,
	\end{array}
\right.
\end{equation*}
is lower semi-continuous if and only if C is closed. 
\begin{proof}
	Take $\xi \in \mathbb{R}$.  If $\xi < 0$, the set $\mu_\xi=\emptyset$. If $\xi >0$, the set $\mu_\xi = C$. Therefore the sets $m_\xi$, for all $\xi \in \mathbb{R}$ is closed if and only if $C$ is closed. By the theorem \ref{th4: levelsets and lsc} $I_C$ is lower semi-continuous if and only if $C$ is closed.
\end{proof}
\end{example}


\paragraph{The Dual Systems of Linear Spaces} 
Two linear spaces $X$ and $Y$ over the same scalar field $\Gamma$ define a dual system if a
fixed bilinear functional on their product is given:
\[(\cdot, \cdot) : X \times Y \rightarrow \Gamma\].

The bilinear functional is sometimes omitted. 

The dual system is called separated if the following two properties hold:
\begin{enumerate}
	\item $\forall x \in X \backslash \{0\}$ there is $y \in Y$ such that $(x, y) \neq 0$.
	\item $\forall y \in Y \backslash \{0\}$ there is $x \in X$ such that $(x, y) \neq 0$. 
\end{enumerate}


In other words, $X$ separates points in $Y$ and $Y$ separates points in $X$.
We consider only separated dual systems.
For each $x \in X$, we define the application $f_x: Y \rightarrow \Gamma$ by
\[f_x(y) = (x, y) \quad \forall y \in Y\]
We observe that $f_x$ is a linear functional on $Y$ and the mapping
$x \rightarrow f_x, \quad \forall x \in X$,
is linear and injective, as can be seen from condition (1). Hence, the correspondence is an embedding. Thus, the elements of $X$ can be identified with the linear
functionals on $Y$. In a similar way, the elements of $Y$ can be considered as linear functionals of $X$, identifying an element $y \in Y$ with $g_y: X\rightarrow \Gamma$, defined by
\[g_y(x) = (x, y), \quad \forall x \in X.\]

Therefore, each dual system of linear spaces defines a mapping from either of the two linear spaces into the space of linear functionals on the other.
We set,
\begin{align*}
p_y(x)=\abs{(x,y)}=\abs{g_y(x)}, & \quad \forall x \in X \\	
q_x(y)=\abs{(x,y)}=\abs{f_x(y)}, & \quad \forall y \in Y
\end{align*}

and we observe that $\mathcal{P} = \{p_y | \ y \in Y \}$ is a family of seminorms on $X$ and $\mathcal{Q}=
\{q_x | \ y \in  X\} $ is a family of seminorms on $Y$. 



\begin{definition}
	If $U$ is a normed space, the the dual space $U^*=\mathcal{B}(U, \mathbb{R})$. Consists of all linear and bonded functionals mapping from $U$ to $\mathbb{R}$.
\end{definition}
\begin{theorem}
	Let be $U$ a Banach space, then the dual $U^*$ is also a Banach space relative to the norm of the functionals defined by \[\norm{u^*}=\sup_{\norm{u}\leq 1} \abs{u^*(u)}\]
\end{theorem}


\begin{remark}
	There is a natural duality between $U$ and $U^*$ determined by the bilinear functional $(\cdot, \cdot): U\times U^*\rightarrow \mathbb{R}$, defined as \[(u, u^*)=u^*(u), \quad \forall u \in U, \, \forall u^* \in U^*\]
\end{remark}

\begin{definition}
	A sequence $(u_n)_n$ in a Banach space is called weakly convergent to some $u\in U$ if for all linears continuous functionals $u^* \in U^*$ we have
	\[
		\lim_{n\rightarrow \infty}u^*(u_n)=u^*(u)
	\]
	$u$ is also called the weak-limit and we write $u_n \xrightharpoonup[n\rightarrow\infty]{} u$.
\end{definition}

\begin{theorem}
	A sequence $(u_n)_n$ in $U$ converges to $u\in U$ if and only if $\sup_{n\in \mathbb{N}} \norm{u_n} < \infty $ and $u_n \xrightharpoonup[n\rightarrow \infty]{}u$
\end{theorem}
\begin{theorem}[Bourbaki-Alaoglu-Katulami]
	The closed unit ball in a Banach space $U$ is weakly compact if and only if $U$ is reflexive. If $U$ is in an addition separable, then it's weakly sequentially compact.
\end{theorem}

\begin{definition}
	Let $U$ be a Banach space and $J: U \rightarrow \mathbb{R}$, $J$ is called weakly (sequentially) lower semi-continuous at point $u_0$ if for every weakly convergent sequence $(u_n)_n$ converges to $u_0$, i.e. $u_n \xrightharpoonup[]{} u_0$, it holds \[(u)\leq \liminf_{n\rightarrow \infty} J(u_n)\]
\end{definition}

\begin{definition}
	A non empty set $C\subset U$ is called weakly closed if for every weakly convergent sequence $(u_n)_n$ in $C$ follows that the weak limit belongs to $C$. i.e. $u_n \xrightleftharpoons[]{} u$, with $u_n \in C$, implies $u\in C$.
\end{definition}

\begin{definition}
	A non empty set $C\subset U$ is called weakly sequentially compact if for every sequence in $C$ contains a weakly convergent subsequence whose limit belongs to $C$.
\end{definition}

\begin{theorem}
	Let $U$ be a Banach space and $J:U \rightarrow \overline{\mathbb{R}}$ the the following conditions are equivalent:
	\begin{itemize}
		\item J is weakly lower semi-continuous on $U$ for all $u\in U$.
		\item The level sets $\mu_\xi = \lbrace u \in U | J(u)\leq \xi \rbrace$ is weakly closed for each $\xi \in \mathbb{R}$.
	\end{itemize}
\end{theorem}

\begin{lemma}
	Let be $J: U \rightarrow \overline{\mathbb{R}}$ a convex and lower semicontinuous functional. Assume there is $u_0 \in U$ such that $J(u_0)=-\infty$, then $J$ is nowhere finite.
	\begin{proof}
		Assume that there is $v\in U$ such that $-infty< J(v)< \infty$. Then by convexity $J(\lambda u_0 + (1-\lambda)v)=- \infty$, $\forall \lambda \in [0,1]$. Because $J$ is lower semicontinuos it follows that in the limit $\lambda \rightarrow 0$, 
		\[
			(\lambda u_0 +(1-\lambda)v) \rightarrow v \implies J(v)\leq J(\lambda u_0 +(1-\lambda)v)=-\infty
		\]
	\end{proof}
\end{lemma}
\begin{lemma}
	Every lower semi-continuous and convex function on a linear space $U$ is weakly lower semi-continuous.
\end{lemma}

\begin{corollary}
	Assume that $U$ is a reflexive Banach space, then every bounded sequence $(u_n)_n \in U$ that is $\sup_{n\in \mathbb{N}} \norm{u_n}<\infty$ has a subsequence $(u_k)_k$ which is weakly convergent to some $u\in U$. 
\end{corollary}

\begin{remark}
	Since every Hilbert space is reflexive the corollary applies to this case.
\end{remark}

\begin{lemma}
	A closed set $C$ is weakly closed if and only if the set is convex.
\end{lemma}

\begin{definition}
	Let $U$ be a real linear space and $J:U\rightarrow \overline{\mathbb{R}}$. We said that $J$ is sublinear if:
	\begin{align*}
		J(\lambda u)=\lambda J(u) & \qquad  \forall u \in U, \text{ and } \mathbb{R}\ni \lambda>0 \\
		J(u+v)\leq J(u)+J(v) & \qquad \forall u, v \in U
	\end{align*}
	\begin{remark}
		Every sublinear function is convex.
	\end{remark}
\end{definition}
\begin{theorem}
	Let $U$ be a real linear space $J: U\rightarrow \overline{\mathbb{R}}$ a sublinear functional. Then there is a linear functional $f$ on $U$ such that,
	\[
		f(u)\leq J(u) \quad \forall u \in U
	\]
\end{theorem}

\begin{definition}
	Let $J:U\rightarrow \overline{\mathbb{R}}$, we said that $J$ is locally bounded around $u_0$ if $\exists V\subset U$ neighborhood of $u_0$ such that for some $M \in \mathbb{R}$
	\[
		\abs{J(u)} < M \qquad \forall u \in V
	\]
\end{definition}
\begin{lemma}
	Let $J:U\rightarrow \mathbb{R}$ convex and $U$ is a Banach space. If $J$ is locally bounded around $u$, then $J$ is lower semi-continuous in $u$.
	\begin{proof}
		Let $u_k \rightarrow u$ as $k\rightarrow\infty$. For each $\epsilon > 0$ we can find a sequence $\alpha_k$ such that $\norm{\frac{u-u_k}{\alpha_k}}< \epsilon$, and $\alpha_k \rightarrow 0$ as $k\rightarrow \infty$. (Please read Maximal Monotone Operators and Evolution Systems
		in Banach Spaces of Barbu. Details Still to be recovered).
		
		Moreover, for $k$ sufficiently large we have $\norm{u-u_k}<\epsilon$. Choose $\epsilon$ such that $J$ is bounded in $\overline{B_{2\epsilon}(u)}$ by $M$ and define $v_k = u_k + \frac{u-u_k}{\alpha_k} \in \overline{B_{2\epsilon}(u)}$, since  $\norm{v_k-u}\leq \norm{u_k-u}+\norm{\frac{u-u_k}{\alpha_k}}\leq 2\epsilon$. 
		
		Since $J$ is convex \[J(u)\leq \alpha_k J(v_k) + (1-\alpha_k)J(u_k) \leq \alpha_k M+J(u_k)\].
		
		Since $\alpha_k \rightarrow 0$, then
		\[J(u)\leq \liminf_{k\rightarrow \infty}(\alpha_kM+J(u_k)) = \liminf_{k\rightarrow\infty} J(u_k) \]
		Thus is if $J$ is convex and locally bounded around $u$, then is lower semi-continuous around $u$.
	\end{proof}
\end{lemma}
\begin{remark}
	The result that convexity and local boundedness imply lower semi-continuity is similar to classical result for linear operators where local boundness implies continuity. In general convexity plays in optimization the same role as linearity in solving equations.
\end{remark}