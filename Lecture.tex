% \documentclass[CEJM,DVI]{cej} % use DVI command to enable LaTeX driver
\documentclass[CEJM,PDF]{cej} % use PDF command to enable PDFLaTeX driver
\usepackage{amsmath, amsthm, amssymb, amsfonts}
\usepackage{graphicx}
\usepackage{layout}
\usepackage{float}
\usepackage{mathtools}
\usepackage{enumitem}
\usepackage{hyperref}   
%%%%%%%%%%%%%%%%%%%%%%%%%%%%%%%%%%%%%%%%%%
%-----------------------------------------------------------------------
\newcommand{\asterisk}[1]{{#1}^{\ast}}
\newcommand{\nabladot}{\nabla\cdot}

\newcommand{\grad}[1]{\nabla\left(#1\right)}
\newcommand{\spdev}[2]{\frac{\partial ^{2} #1}{{\partial #2}^{2}}}
\newcommand{\ppdev}[2]{\frac{\partial ^{2} #1}{{\partial #2}^{2}}}
\newcommand{\pdev}[2]{\frac{\partial ^{2} #1}{{\partial #2}^{2}}}
\newcommand{\pdxo}{\partial_{\asterisk{x}}}
\newcommand{\pdyo}{\partial_{\asterisk{y}}}
\newcommand{\pddxo}{\partial_{\asterisk{x}}^2}
\newcommand{\pddyo}{\partial_{\asterisk{y}}^2}

\newcommand\norm[1]{\left\lVert#1\right\rVert}
\newcommand\argmin[2]{
{\displaystyle {\underset {#1}{\operatorname {arg\,min} }}\left[#2\right]}}
%----------------------------------------------------------------------



%---------------------------------------------------------
\graphicspath{{Figures/}}

%%%%%%%%%%%%%%%%%%%%%%%%%%%%%%%%%%

\title{Optimization}
\shortjournalname{Optimization Notes}
\journalname{Hamburg University}
\articletype{Notes} % Research Article, Review Article, Communication, Erratum




\author{%Oscar Ramirez\inst{1}\email{oscar.ramirez@mathmods.eu}
     }
%\author{First~Author\inst{1}\email{email@first.author.com},
%	Second~Author\inst{2}\email{email@second.author.com}
%}
%\shortauthor{F. Author, S. Author}

\institute{Department of Mathematics, Hamburg University, Bundesstrasse 55 , 20146, Hamburg, Germany}

\abstract{}

\keywords{Optimization\*\ Convexity}

%\msc{XXXXX, YYYYY}

\begin{document}
\maketitle

%\baselinestretch{2}
\section*{Introduction}





\subsection{Basic Definitions}

%----------------------------------------------------------

Let $X$ be a real  vector space, let $C$ and $D$ be subsets of $X$, and let $z \in X$. Then we use the following notation:
\begin{itemize}
	\item $C + D =\lbrace x + y \mid x \in C, y \in D\rbrace$.
	\item $C -D =\lbrace x - y \mid x \in C, y \in D\rbrace$.
	\item $z+C=\lbrace z \rbrace +C$.
	\item  $C-z=C-\lbrace z \rbrace$. 
	\item For every $\lambda \in \mathbb{R}$, $\lambda C =\lbrace\lambda x \mid x \in C\rbrace$.
	\item  If $\Lambda$ is a non-empty subset of $\mathbb{R}$, then $\Lambda C = \bigcup_{\lambda \in \Lambda}\lambda C$, and $\Lambda z = \Lambda \lbrace z \rbrace =\lbrace\lambda z \mid \lambda \in \Lambda$.
\end{itemize}
 
%-------------------------------------------------------------------------------

\begin{definition}[Cone]
	Using the above notation, we define $C$ as a cone if $C=\mathbb{R}^+C$.
\end{definition}

\begin{definition}
	Let $C$ be a subset of $U$ Banach space. The conical hull of $C$ is the intersection of all the cones in $U$ containing $C$, i.e., the smallest cone in $U$ containing $C$. It is denoted by $\cone C$. The closed conical hull of $C$ is the smallest closed cone in $U$ containing $C$. It is denoted by $\overline{\cone} C.$
\end{definition}

\begin{definition}
	We say a functional $J$ is proper if $\dom J \neq \emptyset$ and $J>-\infty$.
\end{definition}

\begin{definition}
	Let $U$ a vector topological space. We define as the indicator function $I_C:U\rightarrow \overline{\mathbb{R}}$ of a set $C \subset U$, as follows:
	\begin{equation*}
		I_C(x)=\left\lbrace
		\begin{array}{ll}
		0	&\quad x \in C \\
		\infty &\quad \text{otherwise.}
		\end{array}
		\right.
	\end{equation*}
\end{definition}

\subsection{Useful lemmas and Theorems.}
\begin{lemma}
	\label{lemma0. Bounded and weakly convergent}
	Let $(x_n)_{n \in \mathbb{N}}$ be a bounded sequence in a Hilbert Space $H$. Then $(x_n)_{n \in \mathbb{N}}$ possesses a weakly convergent subsequence.
\end{lemma}

\begin{lemma}
Let $(x_n)_{n \in \mathbb{N}}$ be a sequence in a Hilbert Space $H$. Then $(x_n)_{n \in \mathbb{N}}$ converges if and only if it is bounded and possesses at most one weak sequential cluster point.
\end{lemma}
\begin{fact}
	A Banach space is reflexive if its unit ball is compact in the weak topology. This implies that every bounded sequence admits a weakly converging subsequence. Hilbert spaces and $L^p$ spaces ($1<p<\infty$) are reflexive.
\end{fact}
\begin{theorem} 
	Let $f: H \rightarrow (−\infty, \infty]$ be a convex functional on a Hilbert space. Then the following are equivalent:
	\begin{enumerate}[label=(\roman{*})]
\item 	$f$ is weakly sequentially lower semicontinuous.
\item 	$f$ is sequentially lower semicontinuous.
\item 	$f$ is lower semicontinuous.
\item 	$f$ is weakly lower semicontinuous.
	\end{enumerate}
\end{theorem}



\begin{lemma}
	A convex set is closed if and only if it is weakly closed.
\end{lemma}

\begin{lemma}
	Every bounded linear operator over a Banach Space is weakly continuos.
\end{lemma}

\begin{lemma}[Parallelogram law]
	
\[
\norm{x+y}^2 +\norm{x-y}^2 = 2\norm{x}^2+2\norm{y}^2
\]
\end{lemma}

\begin{lemma}
	Let $\mathcal{X}$ be a Hausdorff space and let $(f_i)_{i \in I}$ be a family of lower semicontinuous functions from $\mathcal{X}$ to $[-\infty, \infty]$. Then $\sup_{i\in I}f_i$ is lower semi-continuous. If $I$ is finite, then $\min_{i\in I}f_i$ is lower-semicontinuous.
\end{lemma}
\begin{definition}
	Let $\mathcal{X}$ be a Hausdorff space. The lower semicontinuous envelope of $f: \mathcal{X}\rightarrow[-\infty, \infty]$ is 
	\[
	\overline{f}=\sup \left\lbrace g: \mathcal{X}\rightarrow[-\infty, \infty] \ |\quad  g\leq f \text{ and } g \text{ is lower semicontinuous}\right\rbrace.
	\]
\end{definition}
\begin{proposition}
	If $C$ is a compact set in a normed space $U$, and $G$ is a closed subset of $C$. Then $G$ is compact.
	\begin{proof}
		Let $\{ g_n\}$ a sequence contained in $G$. Since $G \subset C$ and $C$ compact. $\exists \{g_n\}_k$ subsequence of $\{g_n\}$, contained in $G$ such that $\{g_n\}_k \rightarrow g$, as $k \rightarrow \infty$, and then since $G$ is closed $g \in G$. Therefore $G$ is compact.
	\end{proof}
	\label{prop0. closed subset compact. Precompact}
\end{proposition}

\begin{definition}
	Let $U$ a vector real space. We denote the set of all 
\end{definition}
\begin{proposition}
	\label{prop0. Convex and increasing derivative}
	Let $\phi:\mathbb{R}\rightarrow (-\infty, \infty]$ be a proper function that is differentiable on a nonempty open interval $C$ in $\dom \phi$. Then the following hold:
	\begin{itemize}
		\item $\phi$ is increasing on $C$. Then $\phi + I_C$ is convex.
		\item Suppose that $\phi'$ is strictly increasing on $C$. Then $\phi$ is strictly convex on $C$. 
	\end{itemize}
\end{proposition}

%%%%% Text taken from Barbu%
% Fulfilling the gaps later.


%The locally convex topology defined
%by $\mathcal{P}$ on $X$ is called the weak topology or $Y-topology$ of X induced by the duality
%$(X,Y)$, and we denote it by $\sigma(X,Y)$. Similarly, the weak topology or X-topology
%of Y, denoted by $\sigma(Y, X)$, is the locally convex topology on Y generated by $Q$.
%Clearly, the roles of $X$ and $Y$ are interchangeable here, since there is a natural duality between $Y$ and $X$ which determines a dual system $(Y,X)$. Thus, it is sufficient to
%establish the properties only for the linear space $X$. According to the well-known re-sults concerning the locally convex topologies generated by families of seminorms,
%we immediately obtain the following result.



\section{Lecture 1}
\subsection{Infinite-Dimensional Optimization} 
Let $(U,d)$ be a metric space and $J:U\rightarrow \overline{\mathbb{R}}$. We call a minimization problem.
\[
	\min_{u\in C} J(u)
\]


\begin{definition}
	A point $u\in U$ is called:
	\begin{itemize}
		\item \textbf{Local Minimizer.} If there is a neighborhood $V \in U$ such that $J(u) \leq J(v)$, $\forall v \in V$.
		\item \textbf{Global Minimizer. } If $J(u)\leq J(v)$, $\forall v \in U$.
	\end{itemize}
\end{definition}
\begin{definition}
	Let be $\{u_k\} \in U$, a convergent sequence in $U$, such that converges to $u \in U$. The functional $J$ is called lower semicontinuous at $u\in U$ if
	\[
		J(u) \leq \liminf_{k\rightarrow\infty} J(u_k).
	\]
	In general if $J$ is lower semicontinuous at $u$, for all the $u \in U$. $J$ is lower semicontinuous (l.s.c). 
\end{definition}

\begin{theorem}
	Let $J: U\rightarrow \mathbb{R}$ lower semicontinuous functional and $\exists \xi \in \mathbb{R}$, such that the level set $\mu_\xi=\{ u \in U \mid J(u)\leq \xi\}$ be non-empty and compact set of $U$. Then there exists a global minimum.
	\begin{proof}
		Let $\alpha := \inf_{u\in U} J(u)$. Then $\exists \{u_n\} \in U $ such that $J(u_n) \rightarrow \alpha$. Then  $\exists N\in \mathbb{N}$, such that $\forall k \geq N$,  $J(u_k) \leq r$ (otherwise $r=\alpha$), then we have since $\mu_\xi$ is not empty, $u_k \in \mu_\xi$. Since $\mu_\xi$ is compact,  $\exists \{u_k\}_l$ a subsequence of $\{u_k\}$ that converges in $\mu_\xi$, i.e. $\{u_k\}_l \rightarrow \overline{u} \in \mu_\xi$, as $l \rightarrow \infty$	. Since $\alpha$ is the infimimum and $J$ is lower semicontinuous and,
		 \[\alpha \leq J(\overline{u}) \leq \liminf_{l\rightarrow \infty} J(u_{k_l}) \]
		 On the other hand,  since $J(u_k) \rightarrow \alpha$,
		 \[
		  \liminf_{l\rightarrow \infty} J(u_k) \leq \alpha
		 \]
		 Therefore $J(\overline{u})=\alpha$, and hence $\overline{u}$ exists and it is a global minimizer.
	\end{proof}
	\label{th1. Existence of global minimizer}
\end{theorem}
\begin{corollary}
	Let $U$ be a Banach space. If the following conditions hold:
	\begin{itemize}
		\item $\exists \mu_\epsilon \in U$ (level set) non-empty and compact.
		\item $J: U \rightarrow \overline{\mathbb{R}}$ is lower semicontinuous. 
	\end{itemize}
	Then set of global minimizers  $G$ is compact.
	\begin{proof}
		The theorem \ref{th1. Existence of global minimizer} implies that all minimizers are in the set $\mu_\xi$. Therefore by proposition \ref{prop0. closed subset compact. Precompact}, $G$ is precompact. Since $J$ is lower semicontinuous, for any convergent sequence $(u_k) \in G$, we have
		\[
			\alpha \leq J(u) \leq \liminf_{k\rightarrow\infty} J(u_k)=\alpha
		\]
		Implying that the limit is also a global minimizer. Hence $G$ is closed.
	\end{proof}
\end{corollary}
\subsection{Derivatives}
Let $U$ and $V$ Banach spaces and $F:U\rightarrow V$ a mapping from $U$ to $V$ (that could be non linear).
\begin{definition}
	Let $C$ be a subset of $U$, let $F: C \rightarrow V$, and let $x \in C$
	be such that, for all $y \in U$, $\exists \alpha >0$ and the set $[x, x + \alpha y] \subset C$. Then $F$ is G\^ateaux
	differentiable at $x$ if there exists an operator $\mathsf{D}F(x) \in \mathcal{B}(U, V)$, called the G\^ateaux derivative of $F$ at $x$, such that,
	\[
	\forall (y \in U)\quad \mathsf{D}F(x) = \lim\limits_{\alpha \downarrow 0} \frac{F(x+\alpha y)-F(x)}{\alpha}
	\]
	
	 Thus, the second G\^ateaux derivative of $F$ at $x$ is the operator $\mathsf{D}^2 F(x) \in \mathcal{B}(U, \mathcal{B}(U, K))$ that
	 satisfies \[(\forall y \in  U) \quad \mathsf{D}^2 F(x) y = \lim\limits_{\alpha \downarrow 0} \frac{\mathsf{D}F(x + \alpha y) - \mathsf{D}F(x)}{\alpha} 
	 \]
\end{definition}
 
\begin{remark}
	The G\^ateaux derivative $\mathsf{D}F(x)$ is unique whenever it exists.	
\end{remark}

\begin{definition}
	Let $x \in U$, let $C$ a set contained  in a neighborhood $\mathcal{V}(x)$ de $x$, and let $F : C \rightarrow V$. Then $F$ is
	Fr\'echet differentiable at $x$ if there exists an operator $\mathsf{D}F(x)\in \mathcal{B}(U, V)$, called the Fr\'echet derivative of $F$ at $x$, such that
	\[
	\lim_{0\neq\norm{y}\rightarrow 0}\frac{\norm{F(x+y)-F(x)-\mathsf{D}F(x)y}}{\norm{y}}=0.
	\]
	
	Higher-order Fr\'chet derivatives are defined inductively. Thus, the second Fr\'echet derivative of $F$ at $x$ is the operator $\mathsf{D}^2 F(x) \in  \mathcal{B}(U, \mathcal{B}(U, V))$ that satisfies,
	\[
		\lim_{0\neq\norm{y}\rightarrow 0}\frac{\norm{\mathsf{D}F(x+y)-\mathsf{D}F(x)-\mathsf{D}^2F(x)y}}{\norm{y}}=0.
	\]
	\label{def1. Frechet.}
\end{definition}


\begin{lemma}
	Let $x \in U$, let be $C$ a set $\mathcal{V}(x)$ contained in a neighborhood of $x$, and let $F: C \rightarrow V$. Suppose that $F$ is Fr\'echet differentiable at $x$. Then the following hold:
	\begin{itemize}
	\item  $F$ is G\^ateaux differentiable at $x$ and the two derivatives coincide.
	\item $F$ is continuous at $x$.
	\end{itemize}
	\begin{proof}
		Denote the Fr\'echet derivative of $F$ at $x$ by $L_x$.
		\begin{itemize}
			\item Let $\alpha >0$ and $y \in U\backslash\{0\}$. Then
			
			\[
				\norm{\frac{F(x+\alpha y)-Fx}{\alpha}-L_xy} = \norm{y}\frac{\norm{F(x+\alpha y)-Fx-L_x(\alpha y)}}{\norm{\alpha y}}
			\]
			converges to $0$ as $\alpha \downarrow 0$, since $F$ is Fr\'echet differentiable.
			\item Fix $\epsilon > 0$. By definition \ref{def1. Frechet.}, we can find $\delta \in  (0, \frac{\epsilon}{\epsilon + \norm{L_x}}]$, such that for all $y$ in the open ball of radius $\delta$ and center in zero, (i.e. $\forall y \in B_\delta(0)$), 
			\[
			   \norm{F(x+y)-Fx-L_xy} \leq \epsilon \norm{y}\]
			Thus, $\forall y \in B_\delta(0)$,  by triangle inequality, 
			 \begin{align*}  
			\norm{F(x+y)-Fx} &\leq \norm{F(x+y) -Fx-L_xy} + 
			\norm{L_xy} \\ &\leq \epsilon \norm{y} + \norm{L_x }\norm{y} \\ &\leq \delta(\epsilon + \norm{L_x})
			\\
			&\leq \epsilon.
			\end{align*}
			It follows that $F$ is continuous at $x$.
		\end{itemize}
	\end{proof}
\end{lemma}



\begin{fact}
	Let $x\in U$,  let $\mathcal{U}$ be a neighborhood of $x$, and let $G$ be a real Banach space, let $F: \mathcal{U}\rightarrow G$ a mapping from $\mathcal{U}$ to $G$, let $\mathcal{V}$ be a neighborhood of $Tx$, and let $R : V \rightarrow K$.
	Suppose that $T$ is Fr\'echet differentiable at $x$ and that $R$ is G\^ateaux differentiable at $Tx$. Then $R\circ T$ is Gate\^aux differentiable at x and $\mathsf{D}(R \circ T )(x) =
	(\mathsf{D}R(Tx))\circ \mathsf{D}T (x)$. If R is Fr\'echet differentiable at x, then so is $R \circ T $.
\end{fact}

\begin{fact}[]
	Let $x \in U$, let $\mathcal{U}$ be a neighborhood of $x$, let $G$ be a real Banach
	space, and let $F: U \rightarrow U$. Suppose that $F$ is twice Fr\'echet differentiable at $x$.
	Then $\forall (y,z) \in U \times U$,  $(\mathsf{D}^2 F (x)y)z = (\mathsf{D}^2 F (x)z)y$.
\end{fact}



\section{Lecture 2}
\subsection{Convexity}
\begin{definition}
	Let U be linear space. A functional $J:U\rightarrow \overline{\mathbb{R}}$ is called convex, if for $t\in[0,1]$ and $u_1, u_2 \in U$.
	\begin{equation}
		J(tu_1+(1-t)u_2)\leq t J(u_1)+(1-t)J(u_2) \label{eq1. Convex functional}
	\end{equation}
	holds such that the right hand sid is well defined.
	\begin{itemize}
	\item 	J is strictly convex if \eqref{eq1. Convex functional} holds strictly for $\forall u_1, u_2 \in U$, $u_1\neq u_2$ and $t \in (0,1)$ with $J(u_1)<\infty$ and $J(u_2)< \infty$.
	\item An optimization problem, 
	\[
		\min_{u \in C} J(u)
	\] is called convex if both $C$ and $J$ are convex.
	\end{itemize}
\end{definition}

\begin{lemma}
	If C and V are convex in U, then
	\begin{itemize}
		\item  $\alpha V = \lbrace w = \alpha v, v\in C \rbrace$ is convex.
		\item $C+V$ is convex.
	\end{itemize}
	\begin{proof}
		
	\end{proof}
\end{lemma}


\begin{lemma}
	Let $V$ be a collection of convex sets in $U$, then $C=\bigcap_{K \in V} K$ is convex.
	\begin{proof}
			If $C = \emptyset$, then $C$ the statement is vacuously true. Consider $C\neq \emptyset$ and $u_1, u_2 \in C$ then $u_1, u_2 \in K$ for all $K \in V$
			\[
			\implies tu_1+(1-t)u_2 \in K, \quad \forall K \in V \implies tu_1+(1-t)u_2 \in \bigcap_{K \in V} K 
			\]
	\end{proof}
\end{lemma}

\begin{lemma}
		Let $C\in U$ convex and $J: C \rightarrow \mathbb{R}$. Define $\alpha=\inf_{u\in C} J(u)$. Then the set $\Psi = {u | J(u)=\alpha}$ is convex, i.e. the solution of 
		\[
			\min_{u \in C} J(u)
		\] 
		is a convex set.
		\begin{proof}
			Let $u_1, u_2 \in \Psi$ and $u_t=tu_1+(1-t)u_2$.Sinc J is convex, it holds that
			$J(u_t)\leq t J(u_t)+(1-t)J(u_t)=\alpha$. Thus $J(u_t)=\alpha$, $\forall t \in [0,1]$. Implying $u_t \in \Psi$ Hence $\Psi$ is convex.
		\end{proof}
\end{lemma}

\begin{lemma}
Let $U$ be linear normed space, and $C\subset U$ a convex set and $J: U \rightarrow \overline{\mathbb{R}}$ convex functional. Let $\overline{u} \in C$ such that
\[
	J(\overline{u}) \leq J(u) \quad \forall u \in B_\epsilon(\overline{u}) \cap C,
\] 
for some ball $B_\epsilon(\overline{u})$ in $U$ with center in $\overline{u}$. Then $J(\overline{u})=\inf_{u\in C} J(u)$.  In other words, the local minimizer of a convex optimization problem is also a global minimizer.
	\begin{proof}
		Let $B_\epsilon(\overline{u})$ be an open neighborhood of $\overline{u}$ with $J(\overline{u})\leq J(u)$ for all $u \in B_\epsilon(\overline{u})\cap C$.
		Take an arbitrary $\asterisk{u} \in C $ and consider $u_t = t \overline{u}+(1-t)\asterisk{u}$. Since $C$ is convex $u_t \in C$.
		
		For some $t\in(0,1)$, $u_t \in B_\epsilon(\overline{u})$.
		
		Thus, \[J(\overline{u})\leq J(u_t) \leq t J(\overline{u})+(1-t)J(\asterisk{u}).\]
		We have $\forall t \in [0, 1]$ that $(1-t)\leq 0$, then
		\[
			(1-t) J(\overline{u}) \leq (1-t) J(\asterisk{u}) \qquad \forall \asterisk{u} \in C
		\]
		Therefore, $\overline{u}$ is a local minimizer for $C$.
	\end{proof}
\end{lemma}
\begin{theorem}
	\label{theorem2. Banach Derivatives}
	Let $U$ is Banach Space, $C \subset U$ convex and $J: C \rightarrow \mathbb{R}$ Gate\^aux differentiable. Consider the minimization problem.
	\[
	 \min_{u\in C} J(u)
	\]
	\begin{enumerate}
		\item Let $\overline{u}$ be a local solution. Then $J'(\overline{u}; u-\overline{u})\geq 0$, $\forall u\in C$.
		\item If $J$ is convex on $C$, then $J'(\overline{u}; u-\overline{u})\geq 0$, $\forall u\in C$ is necessary and sufficient for global optimality of $\overline{u}$
		\item If $J$ is strictly convex on $C$, then the minimization problem admits at most one solution.
		\item If $C$ is closed, and $J$ is convex and continuous with
		\[
			\lim\limits_{\substack{u \in C\\ \norm{u}\rightarrow \infty }}  J(u) = \infty.
		\] 
		Then a global solution $\overline{u} \in C$ exists.
	\end{enumerate}
\begin{proof}
	\
	\begin{enumerate}
		\item Let $\overline{u}$ be a local solution $J(\overline{u})\leq J(u)$, $\forall u \in B_\epsilon (\overline{u}) \cap C$, let $t\in[0,1]$, $u_t=\overline{u}+t(u-\overline{u})$, then $u_t \in C$, since $C$ is convex. 
		
		For small $t>0$, 
		\[
		 0\leq \frac{1}{t}\left[J(u_t)-J(u)\right]\leq\frac{1}{t}\left[J(\overline{u}+t(u-\overline{u}))-J(u)\right]\xrightarrow{t \downarrow 0} J'(\overline{u}; u-\overline{u})
		\]
		
		\item Since $J$ is convex we have for $u\in C$, $J(\overline{u}+t(u-\overline{u})) \leq J(\overline{u})+t\left[J(u)-J(\overline{u})\right]$, for $t>0$
		
	\[
		\implies 	J(u)-J(\overline{u})\geq\frac{1}{t}\left[J(\overline{u}+t(u-\overline{u}))-J(\overline{u})\right] \xrightarrow{t \downarrow 0} J'(\overline{u}; u-\overline{u}) \geq 0.
	\]
	Therefore $\overline{u}$ is a global minimizer.
	
	\item Assume, that there are two solution for the minimization problem, $\overline{u}, \asterisk{u} \in C$, such that $\overline{u}\neq \asterisk{u}$ and $J(\overline{u})=J(\asterisk{u})=\inf_{u\in C} J(u)$. Since $J$ is strictly convex $J(u_t)=J(t\overline{u}+(1-t)\asterisk{u})<tJ(\overline{u})+(1-t)J(\asterisk{u}) = \alpha$ for all $t \in [0,1]$. Contradicting our assumption that $\asterisk{u}$ and $\overline{u}$ are solutions.
	\item $\alpha = \inf_{u \in C} J(u) \in \mathbb{R}\cup\{-\infty\}$, choose a minimizing sequence $(u_k)_k\subset C$ with $J(u_k)\xrightarrow{k\rightarrow \infty} \alpha$
\begin{align*}
	\implies& (u_k)_k \mbox{ is bounded, because } J \rightarrow \infty \mbox{ as } \norm{u}\rightarrow \infty. \\
	\implies& (u_k)_k \mbox{ contains a weakly convergent subsequence } u_{k_e} \xrightharpoonup[e\rightarrow \infty]{}\overline{u}\in C. \mbox{ Since C is closed and convex.}\\
	\implies& J \mbox{ is weakly-lower semicontinuos because it is convex and continuos.}
\end{align*}

	\end{enumerate}
\end{proof}	

\end{theorem}
\section{Lecture 3}
Now consider Hilbert space $(H, \langle \cdot,\cdot \rangle)$ with the norm defined as $\norm{\cdot}=\sqrt{(\cdot,\cdot)}$.

Let be $J: H\rightarrow \mathbb{R}$ a functional over a Hilbert space $H$, we define the set,
\[
{\displaystyle {\underset {v\in C\subseteq H}{\operatorname {arg\,min} }}\,J(x):=\{x\mid x\in C\wedge \forall v\in C:\,J(x)\leq J(v)\}.}\]


By Riesz-Fr\'echet representation formula, exists a unique vector $\nabla J(x) \in H$ such that, 
\[
(\forall y \in H) \quad J'(x; y)=\langle y, \nabla J(x)\rangle
\]
namely Gate\^aux gradient of $J$ at $x$. 

\begin{lemma}
\label{lemma3. Projection}
Let $H$ Hilbert space and $C\subset H$ closed and convex. Define $P_C: H\rightarrow C$, \[P_C(x)=\argmin{v\in C}{\norm{v-x}}.\]
Then,
\begin{enumerate}
	\item $P_C$ is well defined, i.e. $\exists !u \in H$ such that $P_C(x)=\{u\}$.
	\item $\forall u,v\in H$, we have $x=P_C(u) \iff x\in C$  and  $\langle u-x, v-x\rangle\leq 0$.
	\item $\norm{P_C(u)-P_C(\overline{u})}\leq \norm{u-\overline{u}}$ \ $\forall u, \overline{u} \in H$, i.e. The projection $P_C$ is non expansive.
	\item $\langle P_C(u)-P_C(\overline{u}), u-\overline{u}\rangle \leq 0, \quad \forall u, \overline{u}\in H$
	\item Let be $t>0$ a real number, then $\forall u \in C$, and $\forall v\in H$, $\phi(t)=\frac{1}{t}\norm{P_C\left(u+tv\right)-u}$ is non-increasing.
\end{enumerate}
\end{lemma}
\begin{theorem}
Let $H$ be Hilbert space, $C \subset H$ closed and convex, $J: C \rightarrow \mathbb{R}$, Gate\^aux differentiable at the local solution $\overline{u}$ of $\min_{u\in C} J(u)$. Thus, $J'(\overline{u};u-\overline{u}) \geq 0$, $\forall u \in C$ and it is equivalent to $ \overline{u}=P_C(\overline{u}-\delta\nabla J(\overline{u})) $
\begin{proof} 
Since every Hilbert Space is a Banach space, and $C$ is closed and Convex subset of $H$, and $\overline{u}$ is a solution of minimization problem; we can apply \ref{theorem2. Banach Derivatives}.\\
 Thus	$J'\left(\overline{u}; u-\overline{u}\right)\geq 0 \iff \langle u-\overline{u}, \nabla J(\overline{u})\rangle \geq 0 \ \forall u \in C$. \\
For all $\delta >0$, we multiply the Gate\^aux gradient  $(-\delta)$ and we have,
\[\langle u-\overline{u}, -\delta \nabla J(\overline{u}) \rangle \leq 0 \ \forall u \in C,\] adding zero to the gradient, $ \langle u-\overline{u}, \overline{u}-\delta\nabla J(\overline{u})-\overline{u}\rangle \leq 0$. Then we set $w \in H$ as $w:= \overline{u}-\delta  \nabla J(\overline{u})$, and applying lemma \ref{lemma3. Projection} we have,
\[
	\overline{u}=P_c(w)	\iff \langle w-\overline{u}, u -\overline{u}\rangle
\]
Thus, 
	\[
		\overline{u}=P_C(\overline{u}-\delta J(\overline{u}))
	\]
\end{proof}
\end{theorem}
\subsection{Application}
Consider $U, Y, Z$ Hilbert spaces. Let be  $J: Y\times U \rightarrow \mathbb{R}$ a functional. Consider the minimization problem,
\begin{equation*}
	\left\lbrace
	\begin{array}{l}
	\overline{u}=\underset{y,u}{\min} J(y,u) \\
	Ay=Bu \quad u \in U_{ad} \subset U
	\end{array}
	\right.
\end{equation*}
For some set $U_{ad}$ closed, convex and bounded. And $A \in \mathcal{L}(Y,Z)$ bounded and invertible and $B\in \mathcal{L}(Y, Z)$.

Then we can write $y \in Y$ as a function of $u \in U$,
\[
y=y(u)=A^{-1}Bu
\]
Consider the reduced cost functional  $	F(u):=J(y(u), u)$, then our problem is equivalent to
\[
	\overline{u}_{ad}=\min_{u \in U_{ad}} F(u) 
\]


Let $(u_k)_k \in U_{ad}$ denote a minimizing sequence, i.e. $F(u_k) \rightarrow \inf_{u\in U_{ad}}F(u)$, since $u_k \in U_{ad}$ the sequence is bounded. Therefore we can find a convergent subsequence $u_{k_{l}} \xrightharpoonup[l\rightarrow \infty]{} \asterisk{u}$, moreover since $U_{ad}$ is closed and convex $U_{ad}$ is weakly closed, implying $\asterisk{u} \in U_{ad}$

\begin{proposition}
	If $J$ is continuous and weakly lower semicontinuous, then 	$\asterisk{u}=\argmin{u\in U_{ad}}{F(u)}$.
	\begin{proof}
		If J is weakly lower semicontinuos 
			\[
				J(y(\asterisk{u}), \asterisk{u})\leq \liminf_{l\rightarrow \infty} J(y(u_k),u_k)
			\]
		That is, 
		\[
			F(\asterisk{u}) \leq \liminf_{l\rightarrow \infty} F(u_k) =\alpha 
		\]
	\end{proof}
\end{proposition}
\section{Lecture 4}
\begin{lemma}
	Let U be linear space and $J: U\rightarrow \overline{\mathbb{R}}$. Then
	\begin{enumerate}
		\item If $J$ is convex, then the effective domain $\mathrm{dom}(J)=\lbrace u\in U | \ J(u) < \infty\rbrace $ is convex.
		\item $J$ is convex $\iff$ $\mathrm{epi}(J)=\left\lbrace (u,\alpha) \in U\times \mathbb{R} | J(u)\leq \alpha\right\rbrace$ is convex.
	\end{enumerate}
	\begin{proof} Since $U$ and $\mathbb{R}$ are linear spaces, is easy to see that scalar multiplications and sums are well defined over $U\times \mathbb{R}$ and so over $\epi(J)$.
		\begin{enumerate}
			\item Assume $J$ convex. If $u_1\in \dom (J)$ and $u_2$ are elements of $\dom (J)$. Therefore, $J(u_1)<\infty$, and $J(u_2)<\infty$, therefore for $t \in [0,1]$, we have $tJ(u_1)<\infty$ and $(1-t)J(u_2)<\infty$. Since $J$ is convex, 
			\[J(t u_1 + (1-t)u_2)\leq tJ(u_1)	+(1-t)J(u_2) < \infty\],
			
			Therefore, $tu_1 +(1-t)u_2 \in \dom (J)$. Hence $\dom J$ is convex.
			\item First consider $J$ a convex functional, then we have for all $u_1, u_2 \in U$,
				\[
					J(t u_1 + (1-t)u_2)\leq tJ(u_1)	+(1-t)J(u_2) \quad \forall t \in [0,1]
				\]
				Let $(u_1, \alpha_1)$, $(u_2, \alpha_2)$ elements of $\epi(J)$, then $J(u_1) < \alpha_1$ and $J(u_2)< \alpha_2$. Since $J$ is convex. 
				\[
					J(t u_1 + (1-t)u_2) \leq t \alpha_1 + (1-t)\alpha_2
				\]
				Then $(tu_1+(1-t)u_2, t \alpha_1+(1-t)\alpha_2) \in \epi(J)$. Therefore, if $J$ is convex, and $(u_1, \alpha_1)$, $(u_2, \alpha_2)$ are elements of $\epi(J)$ then,
				\[ t(u_1,\alpha_1)+(1-t)(u_2, \alpha_2)=(tu_1+(1-t)u_2, t \alpha_1+(1-t)\alpha_2) \in \epi(J) \]
				Hence $\epi(J)$ is convex. 
				
				Now assume $\epi(J)$ convex. Let $(u_1, \alpha_1)$, $(u_2, \alpha_2)$ elements of $\epi(J)$ then $(tu_1+(1-t)t \alpha_1 + (1-t)\alpha_2) $, then
				\[
							J(t u_1 + (1-t)u_2) \leq t \alpha_1 + (1-t)\alpha_2 \quad \forall t \in [0,1]
				\]
				
				By definition of $\epi(J)$, if $u_1, u_2 \in \dom J$, then $(u_1, J(u_1))$ and $(u_2, J(u_2))$, are elements of $\epi(J)$, therefore
				\[
					J(t u_1 + (1-t)u_2) \leq t J(u_1) + (1-t)J(u_2) \quad \forall t \in [0,1]
				\]
				Implying that $J$ is convex.
		\end{enumerate}
	\end{proof}
\end{lemma}

\begin{definition}
	
\end{definition}


\section{Lecture 5}
\subsection{Subgradients}
\begin{proposition}
	 Let $J: U \rightarrow (-\infty, \infty]$ be proper. Suppose that $\text{dom} f$
	 is open and convex, and that $f$ is G\^ateaux differentiable on $\text{dom} f$ . Then the
	 following are equivalent:
	 \begin{itemize}
		 \item $f$ is convex.
		 \item $(\forall x \in \dom f )(\forall y \in \dom f )\quad (x-y\mid \nabla f(x))_{UU^*}\leq f(x)-f(y)$.
		 \item $(\forall x \in \dom f )(\forall y \in \dom f )\quad 0 \leq (x-y\mid \nabla f(x)-f(y)_{UU^*}$, i.e. $\grad f(x)$ is monotone.
	 \end{itemize}
	 \begin{proof}
	 
	 \end{proof}
\label{prop5. First derivative convex.}
\end{proposition}

\begin{theorem}
	\label{th5. Secon derivative convex.}
	 Let $J: U \rightarrow (-\infty, \infty]$ be proper. Suppose that $\text{dom} f$
	 is open and convex, and that $f$ is twice G\^ateaux differentiable on $\text{dom} f$ . Then,
	 \[
		 (\forall x \in \dom f )(\forall y \in \dom f ) \quad (z\mid  \nabla^2 f(x)z)_{UU^*} \geq 0
	 \]
\end{theorem}
\begin{definition}
	Let $U$ be a Banach space and let $J:U\rightarrow (-\infty, \infty]$ be a convex and proper function.
	The subdifferential at a point $u \in \dom J$ is a mapping,
	\[
		\partial J : U \rightarrow 2^{U^*}, \qquad \partial J(u):=\lbrace p^* \in U^* \ | \ J(v)\geq J(u)+p^*(v-u),
		\ \forall v \in U\rbrace
	\]
	The elements of $p^* \in \partial J(u)$ are called subgradients of $J$ at $u$.
\end{definition}
\begin{example}
	Consider $J:\mathbb{R}\rightarrow \mathbb{R}$, $u\rightarrow \abs{u}$ which is not differentiable at $u=0$. If $u>0$, then $J(u)=u$ and we can find $0<v<u<w$. Then $p^* \in \partial J(u)$ implies by definition of subdifferential
	\begin{align*}
		v-u \geq p^*(v-u) &\equiv (1-p^*)(u-v) \leq 0 \\
		w-u \geq p^*(w-u) &\equiv (1-p^*)(w-u) \geq 0.
	\end{align*}
	which implies for $u>0$, $p^* \leq 1 \leq p^*$, then $p^*=1$.
	
	In the same way we obtain for $u<0$, $p^*\geq -1 \geq p^*$. 
	In the case $u=0$, we need to satisfy $\abs{v}\geq p^*v$, which is fulfilled if and only if $\abs{p^*}\leq 1$. Hence for $J(u)=\abs{u}$,
	\begin{equation*}
	\partial \abs{u}=
	\left\lbrace
		\begin{array}{lr}
		\{1\}, & u>0 \\
		\left[-1,1\right], & u=0 \\
		\{-1\}, & u < 0
		\end{array}
	\right. .
	\end{equation*}
\end{example}

\begin{example}
	A convex function which is not subdifferentiable everywhere $J:\mathbb{R} \rightarrow \mathbb{R}$, 
	\begin{equation*}
		J(u)=\left\lbrace\begin{array}{cl}
		 -\sqrt{1-\abs{u}^2} &\quad \abs{u}\leq 1 \\
		 \infty &\quad \text{otherwise}
		\end{array}
	\right.
	\end{equation*}
	For $\abs{u}\geq 1$, we have $\partial J(u) = \emptyset$.
\end{example}

\begin{example}
	Let $C$ be a convex and closed subset of $U$ and $I_C$ function defined by
	\[
		I_C (u)=
		\left\lbrace
		\begin{array}{cl}
			0 \quad& u\in C \\
			\infty \quad& \text{otherwise}
		\end{array}
		\right.
	\]
	The subdifferentiable is the definition of normal cone at $u$
	\[\partial I_C (u)=\lbrace u^* \in U^* \ | \ u^*(u-v)\geq \forall v \in C\rbrace = \mathcal{N}_C(u)\].

\end{example}

\begin{theorem}
	Let $U$ be a Banach space. And $J: U\rightarrow \overline{\mathbb{R}}$ a subdifferentiable function. Then $\partial J(u)$ is convex and weakly closed.
\end{theorem}

\begin{remark}
	Most of the rules for derivates also hold for subdifferentials with some additional assumptions,
	\begin{itemize}
		\item $J:U\rightarrow \overline{\mathbb{R}}$, $\lambda > 0$, $\partial J(\lambda u)=\lambda J(u)$.
		\item $\partial(J+F)(u) \supseteq \partial J(u)+ \partial F(u)$.
	\end{itemize}
\end{remark}

\begin{theorem}[Rockafeller]
	Let $U$ be a Banach space and $J: U\rightarrow \mathbb{R}$ proper and convex functions for $i=1,\dots, n$.
	The sum-rule
	\[
		\partial(J_1+\dots+J_n)(u)=\partial J_1(u)\dots \partial J_n(u), \qquad n\geq 2
	\]
	holds if there exists $u_0 \in U$ such that all $J_i(u_0)$ are finite and all $J_i$ except at most one $J_k$, $k\in \lbrace 1,2,\dots n\rbrace$ are continuous at $u_0$
\end{theorem}
\section{Lecture 6}
\begin{theorem}
	Let $V, U$, Banach Spaces. Let $J:V \rightarrow \overline{\mathbb{R}}$ a convex functional. And consider the mapping $A: U\rightarrow V$ linear and continuous with the adjoint $A^*:U^*\rightarrow V^*$. Moreover, $J$ is lower semi-continuous and let $A\overline{u}$ be a point where $J$ is continuous and finite. Then te compose function $J\circ A: U\rightarrow \overline{\mathbb{R}}$ is subdifferentiable for all $u\in V$ and, 
	\[
		\partial(J\circ A)(u)=A^*\left(\partial J(Au)\right)
	\]
	\begin{proof}
		Let $p^*\in \partial J(Au)$,
		\[
			J(p)\geq J(Au)+p^*(p-Au) \quad \forall p \in V 
		\]
		where $p=Av$ with $v\in U$,
		\begin{align}
			(J\circ A) (v) &\geq (J\circ A) (u)+p^*(A(v-u)) \quad \forall v \in U \\
						   &=(J\circ A)(u)+A^*p^*(v-u) \quad \forall v \in U
		\end{align}
		
		i.e. $A^*p^* \in \partial (J\circ A)(u) \implies A^* \partial J(Au)\subseteq \partial (J\circ A)(u)$. Proof based again on the weak separation theorem of convex sets. (We have to check Bauschke)
	\end{proof}
\end{theorem}

\begin{theorem}
	If $J: U\rightarrow \overline{\mathbb{R}}$ is convex and Frech\'et-differentiable at $u\in U$, then $\partial J(u)=\lbrace J'(u)\rbrace$
	
	\begin{proof}
		Let $p^* \in \partial J(u)$. Then for each $t>0$, $J(u+tv)-J(u)\geq p^*(tv)=tp^*(v)$, diving by $t$ and takin the limit $t\rightarrow 0$ we obtain,
		\begin{align}
			&J'(u)(v) \geq p^*(v) \quad \forall v \in U \\
	\implies& (J'(u)-p^*)(v) \geq 0 \quad \forall v \in U.
		\end{align}
	Since $J'(u)$ is Frech\'et differentiable the operator $J'(u)$ is linear with respect to $v$ and $p^* \in U^*$ implies $(J'(u)-p^*)$ is linear, taking $-v \in U$, we obtain that $(J'(u)-p^*)(v)\leq 0$. Therefore $p^*=J'(u)$.
	
	On the other hand, if $J$ is differentiable, it follows that $J'(u)\in \partial J(u)$. For $v\in U$, we set $w=v-u$, $u\in U$ we have, 
	\begin{align}
		    J(u+w)-J(u)&\geq (J'(u))(w) \\
	\implies \ J(v)-J(u)&\geq (J'(u))(v-u)
	\end{align}
	
	Since the above inequality holds for all $v \in U$ implies $J'(u) \in \partial J(u)$.
	\end{proof}
\end{theorem}
\begin{remark}
	The subgradient can be used to obtain local optimality conditions that are necessary and sufficient for convex problem.
\end{remark}
\begin{theorem}
	Let $U$ be a Banach Space and $J: U \rightarrow \mathbb{R}$ convex and proper. Then each local minimum is global minimum.
	Moreover $\overline{u}\in U$ is a minimizer if and only if $0 \in \partial J(\overline{u})$.
	\begin{proof}
		If $0 \in \partial J(\overline{u})$: $J(v) \geq J(\overline{u})+ (0)(v-\overline{u})=J(\overline{u})$, $\forall v\in U$, and hence $\overline{u}$ is a global minimizer. 
		
		
		Assume that $0 \notin J(\overline{u})$, then $\exists v \in U$, such that
		\[
			J(v) < J(\overline{u}) + (0)(v-\overline{u}) = J(\overline{u}).
		\]
		Therefore $\overline{u}$ cannot be a minimizer.
	\end{proof}
\end{theorem}

\begin{definition}[Duality]
Let $J:U\rightarrow \overline{\mathbb{R}}$, and $U$ a Banach space. Then the convex conjugate function $J^*: U^* \rightarrow \mathbb{R}$ is defined by
\[
	J^*(p^*)=\sup_{u\in U} \left\lbrace p^*(u)-J(u)\right\rbrace
\]
implies that $-\sup_{u\in U} \left\lbrace p^*(u)-J(u) \right\rbrace= -J^*(p^*)= \inf_{u\in U} \left\lbrace J(u)-p^*(u)\right\rbrace$.
\end{definition}

\begin{example}
	Consider the indicator function of a convex set $C$, $I_C: U\rightarrow\overline{\mathbb{R}}$
	\[
	I_C(u)=
	\left\lbrace
	\begin{array}{cl}
	0&\quad u \in C\\
	\infty & \quad \text{otherwise}.
	\end{array}
	\right.
	\]
	
	Then we have that the convex conjugate is given by
	\[J^*(p^*)=\sup_{u\in U}\left\lbrace p^*(u)-I_C(u)\right\rbrace=\sup_{u\in C}\left\lbrace p^*(u)\right\rbrace.\]
\end{example}
\begin{example}
	$J: \mathbb{R}_+\rightarrow \mathbb{R}$
\end{example}
\begin{example}
	Let $J:\mathbb{R}\rightarrow \mathbb{R}$, such that $J(u)=\exp u$, then $J^*(p^*)=\sup_{u\in \mathbb{R}} \lbrace p^*u -\exp u\rbrace$. Let $f(u)=p^*u-\exp(u)$, therefore $f'(u)=p^*-\exp u$, $\forall u \in \mathbb{R}$. Which is zero for $\overline{u} = \ln p^*$, if $p^*>0$. Since $f''(u)<0$, then $\overline{u}$ is indeed maximum. And we see that $\lim\limits_{u\rightarrow \pm \infty} f(u)=-\infty$. If $p^*=0$, $f(u)=-\exp u < 0$ and therefore the $\sup_{u \in \mathbb{R}} f(u)=0$ (Consider the limit when $u \rightarrow -\infty$). Then we have,
	\begin{equation*}
		J^*(p^*)=\left\lbrace 
		\begin{array}{cr}
			p^*(\ln p^*-1) & \quad p^*>0 \\
			0 & \quad p^*=0
		\end{array}		
		\right.
	\end{equation*}	
\end{example}
\begin{example}
	Let $H$ be a Hilbert space and $J(u)=\frac{1}{2} \norm{u}^2$.
	Since $H$ is Hilbert, by Riesz, for each linear and bounded functional $\phi_{p^{*}} \in H$, $\exists p^* \in H$ such that, $\phi_{p^{*}}(u)=\langle u,p^*\rangle$. Using the definition of conjugate function,
	\begin{align*}
	J^*(p^*) &= \sup_{u \in U} \left\lbrace\langle u, p^*\rangle -\frac{1}{2} \norm{u}^2\right\rbrace \\
	&= -\inf_{u \in U} \left\lbrace \frac{1}{2} \norm{u}^2-\langle u, p^*\rangle \right\rbrace
	\end{align*}
Note that,
\[
	\frac{1}{2}\norm{u-p^*}^2=\frac{1}{2}\norm{u}^2-\langle u ,p^*\rangle + \frac{1}{2}\norm{p^*}^2 
\]
Therefore we can substitute in the above equation to find an equivalent form to the conjugate function,
\begin{align*}
J^*(p^*) &=-\inf_{u \in U} \left\lbrace \frac{1}{2} \left(\norm{u-p^*}^2 - \norm{p^*}^2\right) \right\rbrace \\
&=-\frac{1}{2} \inf_{u \in U} \left\lbrace  \norm{u-p^*}^2\right\rbrace +\frac{1}{2} \norm{p^*}^2
\end{align*}
We have $\norm{u-p^*} \geq 0$, $\forall u \in H$, then,
\[
\inf_{u \in U} \left\lbrace  \norm{u-p^*}^2\right\rbrace=0,
\]
since we can take $u=p^*$. Hence,
\begin{equation}
	J^*(p^*)=\frac{1}{2}\norm{p^*}^2
\end{equation}
\end{example}
\begin{theorem}
	Let $U$ be a Banach space and $J:U\rightarrow \overline{\mathbb{R}}$. Then $J^*$ is convex.
	\begin{proof}
		Let $p^*, q^* \in U^*$, and $\lambda \in [0,1]$,
		\begin{align}
			J^*(\lambda p^* +(1-\lambda)q^*) &=\sup_{u\in U}\lbrace (\lambda p^* +(1-\lambda)q^*)(u)-J(u) \rbrace \\
										 &= \sup_{u \in U} \left\lbrace  \lambda p^*(u)	- \lambda J(u) +(1-\lambda)q^* (u)-(1-\lambda)J(u)\right\rbrace\\
										 &\leq \sup_{v,u\in U} \left\lbrace  \lambda p^*(u)	- \lambda J(u) +(1-\lambda)q^* (v)-(1-\lambda)J(v)\right\rbrace \\
										 &=\sup_{u\in U} \left\lbrace  \lambda p^*(u)	- \lambda J(u)\right\rbrace +\sup_{v \in U}\left\lbrace(1-\lambda)q^* (v)-(1-\lambda)J(v)\right\rbrace \\
										 &=\lambda J^*(p^*)+(1-\lambda) J^*(q^*).
		\end{align}	
		Hence $J^*$ is convex.
	\end{proof}
\end{theorem}
\section{Lecture 7}
\begin{remark}
Some elementary properties of conjugate functions
\begin{itemize}
	\item  \textbf{Young inequality} $J(u)+J^*(p^*)\geq p^*(u) \quad \forall u \in U, \forall p^* \in U^*$
	\item  $J^*(0)=\sup_{u\in U}\left((0,u)-J(u)\right)=\sup_{u\in U}\left(-J(u)\right)=\inf_{u\in U}J(u)$
\end{itemize}
In many applications in optimization, is used the equivalent formulation,
\[
\inf_{u\in U}J(u)=-J^*(0).
\]
\end{remark}
$J\leq F \implies J^*\geq F^*$

\begin{theorem}
	Let $U$ a Banach space and $J^*: U^* \rightarrow \overline{\mathbb{R}}$ be the conjugate of the $J:U\rightarrow \overline{\mathbb{R}}$. Then for all $u\in U$. \[p^* \in \partial J(u) \iff J(u)+J^*(p^*)=p^*(u)\]. \label{th7: conjugate plus functional}
	\begin{proof}
		content...
	\end{proof}
\end{theorem}
\begin{corollary}
	It follows from previous theorem that $\partial J(u)=\lbrace p^* \in U^* | J(u)+J^*(p^*)=(p^*, u)\rbrace$.
\end{corollary}
\begin{theorem}
	Let $U$ be a Banach space and $J:U\rightarrow \mathbb{R}$ be proper function. If $p^* \in \partial J(u)$ then $u\in \partial J^* (p^*)$
	\begin{proof}
		Let $p^*\in \partial J(u)$. For any $g^* \in U^*$, it follows \[J^*(g^*)=\sup_{v\in U} \left\lbrace
		g^*(v)-J(v)\right\rbrace\geq g^*(u)-J(u)\geq g^*(u)-J(u)\]
		From theorem \ref{th7: conjugate plus functional} 
		\[
		J^*(g^*)\leq g^*(u)-p^*(u)+J^*(p^*) = \left(g^* - p^* \right)(u)+J^*(p^*)\implies u\in \partial J^*(p^*).
		\]
	\end{proof}
\end{theorem}
By iteration the definition, we obtain the bipolar function
$(J^*)^*=J^{**}: U^{**}\rightarrow \overline{\mathbb{R}}$, 
\[
	J^{**}(u)=\sup_{p^*\in U^*} \lbrace p^*(u)-J^*(p^*) \rbrace
\]

\begin{theorem}[Convex envelope theorem.]
	Let $U$ be a reflexive Banach space. The $J^{**}$ is the maximum convex functional below $J$ (also called convex envelope). That is, if $F$ is convex and $F(u)\leq J(u)$, $\forall u \in U$. Then, $J^{**}(u)\leq J(u)$, and $F(u)\leq J^{**}(u)$, $\forall u \in U$. In particular $J^{**}=J$ if and only if $J$ is convex.
	\begin{proof}
		Let $\phi_u \in U^{**}$. Since $U$ is reflexive every $\phi_u \in U^**$ can be related with a member $u \in U$, defining $\phi_u(p^*)=p^*(u)$, we proceed 
		\begin{align}
			J^{**}(u)&=\sup_{p^*\in U^*} \left\lbrace \phi_u(p^*)-J^*(p^*) \right\rbrace\\
					&=\sup_{p^*\in U^*} \left\lbrace p^*(u)-J^*(p^*) \right\rbrace\\
					&=\sup_{p^*\in U^*} \left\lbrace p^*(u)-\sup_{v\in U}\left\lbrace p^*(v)-J(v)\right\rbrace\right\rbrace\\
					&=\sup_{p^*\in U^*} \left\lbrace p^*(u)+\inf_{v\in U}\left\lbrace J(v)-p^*(v)\right\rbrace\right\rbrace\\
					&=\sup_{p^*\in U^*} \left\lbrace \inf_{v\in U}\left\lbrace p^*(u)+ J(v)-p^*(v)\right\rbrace\right\rbrace\\
					&=\sup_{p^*\in U^*} \left\lbrace \inf_{v\in U}\left\lbrace p^*(u-v)+J(v)\right\rbrace\right\rbrace
		\end{align}
		Taking $v=u$ in the expression and comparing it with its infimum the inequality holds,
			\begin{align*}
				\inf_{v\in U}\left\lbrace p^* \left(u-v\right)+J(v) \right\rbrace &\leq p^*(u-u)+J(u) \\
				\inf_{v\in U}\left\lbrace p^* \left(u-v\right)+J(v) \right\rbrace &\leq J(u)
			\end{align*}
		We have that $J^{**}(u)\leq J(u)$.
		\begin{align*}
		\sup_{p^* \in U^*}\inf_{v\in U}\left\lbrace p^* \left(u-v\right)+J(v) \right\rbrace &\leq J(u)\\
		J^{**}(u) &\leq J(u)
		\end{align*}
		Now we assume that $F$ is a convex functional and $g^* \in \partial F(u)$ for $u \in U$.
		\begin{align}
			\implies  F(v)&\geq F(v)+q^*(v-u)\\
					  F^{**}(u)& = \sup_{p^* \in U^*} \inf_{v \in U} \left\lbrace p^*(u-v)+F(u)+q^*(v-u)\right\rbrace \\
					  &\geq \sup_{p^* \in U^*} \inf_{v \in U} \left\lbrace \left(p^*-q^*\right)(u-v)+F(u)\right\rbrace\\
					  &\geq \inf_{v \in U} \left\lbrace (q^*-q^*)(u-v)+F(u) \right\rbrace\\
					  &=F(u)
		\end{align}
	If $F$ is convex,
	\begin{equation}
		\implies F(u)\leq F^{**}(u)\leq F(u) \implies F(u)=F^{**}(u),
	\end{equation}

	\begin{equation}
	F(u)=F^{**}(u)=\sup_{p^* \in U^*} \inf_{v \in U} \left\lbrace p^*(u-v)+F(v)\right\rbrace \leq J^{**}(u)
	\end{equation}
	\end{proof}
\end{theorem}

\section{Lecture 8}
\begin{definition}
Let $U$ and $Y$ Banach spaces and $J: U\rightarrow \overline{\mathbb{R}}$ is a proper function. We consider an optimization problem \eqref{eq8. P} called primal problem
 \[
 \inf_{u\in U} J(u) \tag{P} \label{eq8. P}
 \] 
 Then the problem is said to be nontrivial if there is $\overline{u} \in U$ such that $J(\overline{u}) < \infty$. A function $\Phi : U \times Y \rightarrow \overline{\mathbb{R}}$ is said to be a perturbation function of $J$, 
 \[
 \inf_{u\in U} \Phi (u,p) \tag{Pp}\label{eq8. Pp}
 \]
 if $\Phi(u, 0)=J(u)$ for all $u \in U$. For each $p \in Y$, the minimization problem \eqref{eq8. Pp} is called a perturbation problem. The variable $p$ is called perturbation parameter. If we denote by $\asterisk{\Phi}$ the convex conjugate function of $\Phi$, the \textit{dual problem}, with respect to $\Phi$ is defined by
 \[
 \sup_{\asterisk{p}\in \asterisk{Y}}-\asterisk{\Phi}(0,\asterisk{p}) \label{eq8. P*}\tag{P*}
 \]
 
 where $\asterisk{\Phi}: \asterisk{(U\times Y)} \cong \asterisk{U}\times\asterisk{Y}\rightarrow \overline{\mathbb{R}}$, a function defined as follows.
 \[
 \asterisk{\Phi}\left(\asterisk{u}, \asterisk{p} \right) = \sup_{\substack{u \in U\\ p\in Y}} \left\lbrace \asterisk{u}(u)+\asterisk{p}(p)-\Phi(u,p)\right\rbrace
 \] 

\end{definition}
 \begin{remark}
For $p=0$, \eqref{eq8. P*} $\equiv$ \eqref{eq8. Pp}. We denote the infimum for problem \eqref{eq8. P} by $\inf\eqref{eq8. P}$ and the supremum for problem \eqref{eq8. P*} by $\sup \eqref{eq8. P*}$
 \end{remark}

 
 \begin{lemma}[Weak duality]
 	For the problem \eqref{eq8. P} and \eqref{eq8. P*} it holds that
 	\[
 	 -\infty \leq \sup\eqref{eq8. P*} \leq \inf\eqref{eq8. P} \leq \infty.
 	\]
 	\begin{proof}
 		Let $\asterisk{p}\in\asterisk{Y}$. It follows
 		\begin{align}
 			-\asterisk{\Phi}(0,\asterisk{p})=&-\sup_{\substack{u\in U\\ p\in Y}}\left\lbrace 0(u) + \asterisk{p}(p)-\Phi(u,p)
 			\right\rbrace\\
 			=&\inf_{\substack{u\in U\\p\in Y}} \left\lbrace\Phi(u,p)-\asterisk{p}(p)\right\rbrace\\
 			\leq & \Phi(u,0)-\asterisk{p}(0)\quad \forall u \in U, \asterisk{p} \in \asterisk{Y}\\
 			\implies \sup_{ \asterisk{p}\in \asterisk{Y}}\left\lbrace-\Phi\left(0, \asterisk{p}\right) \right\rbrace \leq&\inf_{u\in U} \Phi (u, 0) = \inf(P)
 		\end{align}
 	 \end{proof}
 	\end{lemma}
 	By iteration we can define, a bidual problem 
 	\[
 	-\sup_{u \in U} \left\lbrace-\asterisk{\Phi}(u,0)\right\rbrace=\inf_{u\in U} \asterisk{\Phi}(u,0) \tag{P**} \label{eq8. BiDual Problem}
 	\]
 	In case the space $U$ is reflexive then $\asterisk{\asterisk{U}}=U$.
 	
 	If the perturbation function $\Phi(u,p)$ is proper, convex and weakly lower semicontinuous. Then $\asterisk{\asterisk{\Phi}}=\Phi$. In this case $\Phi (u, 0) =\asterisk{\asterisk{\Phi}}(u, 0)$ i.e \eqref{eq8. P}$\equiv$ \eqref{eq8. BiDual Problem}

  \begin{definition}
  		Consider the infimal value function
  		\[
  		h(p)=\inf \eqref{eq8. Pp} = \inf_{u \in U} \Phi(u,p)
  		\]
  	The problem \eqref{eq8. P} is called stable if $h(0)$ is finite and its sub-differentiable in zero is not empty.
  	\end{definition}
  	
  \begin{theorem}
  	The primal problem \eqref{eq8. P} is stable 
  	if and only if the following conditions are simultaneously satisfied:
  	\begin{itemize}
  		\item The dual problem \eqref{eq8. P*} has a solution.
  		\item There is no duality gap, i.e.
  		\[ 
	  		\inf\eqref{eq8. P}= \sup\eqref{eq8. P*} \leq\infty
  		\]
  	\end{itemize}
 \end{theorem}	

\begin{theorem}[Extremal relation]
	Let $\Phi:U\times Y \rightarrow \overline{\mathbb{R}}$, be convex the the following statements are equivalent:
	
	\begin{enumerate}[]
		\item \eqref{eq8. P} and \eqref{eq8. Pp} have solutions $\overline{u}$ and $\overline{\asterisk{p}}$ and $\inf(P)=\sup(\asterisk{P})$
		\item $\Phi(\overline{u}, 0)+\asterisk{\Phi}\left(0, \overline{\asterisk{p}}\right)=0$
		\item $\left(0, \overline{\asterisk{p}}\right) \in \partial \Phi (u,0)$ and $ \left(\overline{u}, 0\right) \in \partial \asterisk{\Phi}(0,\asterisk{p})$
	\end{enumerate}
	\begin{proof}
		We proceed by parts:
		\begin{enumerate}
			\item (1)$\implies$ (2): $\overline{u}$ solution of $\inf\eqref{eq8. P}$ and $\overline{\asterisk{p}}$ solution of $\sup\eqref{eq8. P*}$ and $\inf\eqref{eq8. P}=\sup\eqref{eq8. P*}$. This properties implies, $
			\Phi(\overline{u},0)=\inf\eqref{eq8. P}=\sup\eqref{eq8. P*}=-\Phi(0,\overline{\asterisk{p}})\implies \Phi(\overline{u}, 0)+\asterisk{\Phi}(0, \overline{\asterisk{p}}) =0$.
			\item (2) $\implies$ (1): $-\asterisk{\Phi}(0, \overline{\asterisk{p}})=\sup\eqref{eq8. P*}\leq \inf\eqref{eq8. P}=\Phi(\overline{u}, 0)=-\asterisk{\Phi}(0, \overline{\asterisk{p}}) \implies \sup\eqref{eq8. P*}=\inf\eqref{eq8. P}$
			\item (2)$\iff$(3): $\Phi(\overline{u},0)+\asterisk{\Phi}(0, \overline{\asterisk{p}})=0=(0, \overline{u}) +(\overline{\asterisk{p}},0)=\left((0, \overline{\asterisk{p}}), (\overline{u},0)\right) \iff (0, \overline{\asterisk{p}}) \in \partial\Phi(\overline{u},0) \ \forall u \in U,  \forall \asterisk{p} \in \partial J(u) \iff J(u)+\asterisk{J}(\asterisk{p})=(\asterisk{p}, u)$
		\end{enumerate}
	\end{proof}
\end{theorem}
\paragraph{Fencel duality.}
Consider the functional $J:U\rightarrow\overline{\mathbb{R}}$, 
\[
	J(u)=F(u)+G(Au)
\]
 with $F:U\rightarrow\overline{\mathbb{R}}$,  G convex function $G: V\rightarrow \overline{\mathbb{R}}$ and $A:U\rightarrow V$ bounded and linear.
 
 We introduce the perturbation $\Phi(u, p) = F(u)+G(Au-p)$. The dual problem is obtained with,
 \[
	 \asterisk{\Phi}(0, \asterisk{p})=\sup_{\substack{u\in U\\ p \in V} }\left( (\asterisk{p},p)-F(u)-G(Au-p)\right)
 \]
 For fixed $u$ we set $q: Au -p$.
 
 \begin{align*}
 \asterisk{\Phi}(0, \asterisk{p})=&\sup_{u\in U} \sup_{p\in V} \left(\left(\asterisk{p}, Au-q\right)-F(u)-G(q)\right)\\
 =&\sup_{u\in U} \sup_{p\in V} \left(\left(\asterisk{A}\asterisk{p}, u\right)-(\asterisk{p}, q)-F(u)-G(q)\right)\\
 =&\sup_{u\in U}  \left(\left(\asterisk{p}, Au\right)-F(u)\right)+\sup_{p\in V}\left((-\asterisk{p},q)-G(q)\right)\\
 =&\asterisk{F}(\asterisk{A}\asterisk{p})+\asterisk{G}(-\asterisk{p})
 \end{align*}
 
\begin{theorem} [Duality]
	The problem \eqref{eq8. P} is stable if and only if are simultaneously fulfilled:
	\begin{itemize}
		\item \eqref{eq8. P*} has a solution. If one of $h(p)$ or $h(q)$ is $\infty$ already holds. Now for $h(p), h(q) < \infty$.
		\item $\inf\eqref{eq8. P} = \sup\eqref{eq8. P*} < \infty$
	\end{itemize} 
\end{theorem}
\section{Lecture 9}
We check the optimality conditions.
\begin{align*}
	0&=\Phi (\overline{u},0)+\Phi^*(u, \overline{p^*}) \\
	&= F(\overline{u})+G(A\overline{u})+F^*(A^*\overline{p^*})\\
	&= [F(\overline{u})+F^*(A^*\overline{p^*})-A^*\circ p^*(u)]+[G(A\overline{u})+G^*(-\overline{p^*})-(-p^*)(A\overline{u})]
\end{align*}

Using Young inequality $J(u)+J^*(u^*)-u^*(u)\geq 0$, $\forall u \in U$, and $\forall u^* \in U^*$, we see that both square brackets are nonnegative; and the sum is zero. Then
\begin{align*}
	F(\overline{u})+F^*(A^*\overline{p^*})=A^*\circ p^*(u) &\implies A^*p^* \in \partial F(\overline{u}) \\
	G(A\overline{u})+G^*(-\overline{p^*})=(-p^*)(A\overline{u}) &\implies -p^* \in \partial G(A\overleftarrow{u})
\end{align*}

$F$, $G$ are convex and locally bounded, one can show that 
$\sup\eqref{eq8. P*}=\inf\eqref{eq8. P}$.
\begin{example}[Denoising with bounded variation.]
\	
Let be $u, v \in L^2(\Omega)$. And let be $g: \Omega\rightarrow \mathbb{R}^n$, such that, $g\in C_0^\infty(\Omega, \mathbb{R}^n)$. Consider the following functional $J: L^2(\Omega) \rightarrow \mathbb{R}$, defined as follows,
	\begin{equation*}
		J(u)=\frac{1}{2} \int_{\Omega}\abs{u(x)-v(x)}^2+\alpha \sup_{\norm{g}\leq 1} \int_{\Omega} u \dive (g) dx
	\end{equation*}
	Also consider the minimization problem \[\min_{u\in BV(\Omega)} J(u),\] 
	restricted to the set of functions with bounded total variations,
	\[BV(\Omega)=\left\lbrace u\in L^1(\Omega) \ | \ V(u, \Omega) < \infty\right\rbrace,\]
	
	where a total bounded variation is defined as,
	\begin{equation*}
		V(u, \Omega)=\sup\left\lbrace 
		 \int_{\Omega} u\ \dive (g) dx;  \ \text{such that} \ g \in C_0^\infty (\Omega, \mathbb{R}^n), \  \norm{g}_\infty \leq 1
		\right\rbrace
	\end{equation*}

	 \begin{remark}
	 For $u$ smooth enough, it is possible to apply integration by parts, considering the contributions due $g$ has compact support and $\Omega\subset \mathbb{R}^n$, $\int_{\Omega} u\ \dive g dx = -\int_{\Omega} g\cdot \nabla u  dx$.
	 \end{remark}

	Consider the norm defined on $BV(\Omega)$ as follows,	
		\[\norm{u}_{BV}:= \norm{u}_{L^1(\Omega)}+V(u, \Omega).\]
	If we consider $J(u)=F(u)+G(Au)$, we can set
	\begin{align*}
		F(u)&=\frac{1}{2}\int_{\Omega} \abs{u(x)-v(x)}^2 dx = \frac{1}{2}\norm{u-v}_{L^2(\Omega)}^2 \\
		G(Au)&=\alpha\int_{\Omega} \abs{\nabla u}dx
	\end{align*}
	 Where $A:=\alpha \nabla$, and $G(u)=\int_{\Omega} \abs{u}dx$. We introduce the convex functional of each function,
	\begin{align*}
		F^*(q^*)&=\frac{1}{2} \int_{\Omega}\abs{q^*(x)-v(x)} ^2 -\frac{1}{2} v^2(x) dx & \quad \forall q^*\in L^2(\Omega)
\\
G^*(p^*) &={\left\lbrace 
	\begin{array}{cl}
		0, &\quad \norm{p^*} \leq 1 \\
	-\infty, &\quad \text{otherwise}
	\end{array}
	\right.}  & \quad \forall p^* \in C_0^\infty(\Omega, \mathbb{R}^n)
	\end{align*}
	
	In order to apply the Fencel duality we see that the , adjoint of $A$ is given by $A^*=-\alpha (\nabla\cdot)$, thus	
	\[-J(p^*)=\frac{1}{2} \int_{\Omega} \abs{-\alpha \nabla\cdot p^* + v^2}^2+ \frac{1}{2} v^2 dx\]
\end{example}

\subsection{Lagrangians}
\begin{definition}
	The function $L:U\times Y^* \rightarrow \overline{\mathbb{R}}$, $-L(u, p^*)=\sup_{p\in Y} \left\lbrace p^*(p) - \Phi(u, p)
	\right\rbrace$, is called Lagrangian of \eqref{eq8. P} relative to the perturbation $\Phi$. If we denote by $\Phi_u$ for fixed $u\in U$ the function $p\rightarrow \Phi(u, p)$, then $-L(u, p^*)=\Phi^*_u(p^*)$
\end{definition}

\begin{lemma}
	For all $u\in U$, the function $L_u: Y^* \rightarrow \overline{\mathbb{R}}$, $p^*\rightarrow L(u, p)$ is  a concave function (i.e. $-L_u$ is convex) and weak upper semi-continuous. If $\Phi$ is convex the for all $p^* \in Y^*$ the function $L_{p^*}: U\rightarrow \overline{\mathbb{R}}$, $u\rightarrow L(u, p^*)$ is convex.
	\begin{proof}

	\end{proof}
\end{lemma}

Without assuming anything about $\Phi$, we obtain
\begin{align*}
	\Phi^*(u^*, p^*) &= \sup_{u\in U, p\in Y} \left\lbrace u^*(u) + p^*(p) - \Phi(u,p) \right\rbrace \\
	&= \sup_{u\in U} \left\lbrace u^*(u) + \sup_{p\in Y}\left[p^*(p) - \Phi(u,p)\right]
	\right\rbrace \\
	&= \sup_{u \in U}\left\lbrace  u^*(u)-L(u, p^*) \right\rbrace
\end{align*}
This implies that,
\[
	\eqref{eq8. P*} \sup_{p^* \in Y^*} \left\lbrace -\Phi^*(0, p^*) \right\rbrace = \sup_{p^* \in Y^*} \inf_{u \in U} L(u, p^*)
\]

Now we assume that $\Phi$ is convex and weak lower semi-continuous, then for $u\in U$, the function 
$\Phi_u : Y \rightarrow \overline{\mathbb{R}}$ is convex and weak lower semi-continuous and thus $\Phi_u^** =\Phi_u$. Moreover
\begin{align*}
	\Phi(u,p)&=\Phi_u^{**}(p) \\
	&=\sup_{p^* \in Y^*}\left\lbrace
	p^*(p)-\Phi_u^*(p)
	\right\rbrace\\ 
	&=\sup_{p^*\in Y^*}\left\lbrace
	p^*(p)+L(u, p^*)
	\right\rbrace \\
	&=\sup_{p^*\in Y^*}\left\lbrace L(u, p^*)\right\rbrace
\end{align*}

Thus,
\begin{equation}
\eqref{eq8. P} \quad \inf_{u, p} \Phi(u, p) = \inf_{u \in U} \sup_{p^*\in Y^*} L(u, p^*)
\end{equation}

\begin{remark}
	The problems \eqref{eq8. P} and \eqref{eq8. P*} are related to min-max problem we have that the weak duality means 
	\[
		\sup \inf L \leq \inf \sup L
	\]
\end{remark}

\begin{definition}
	An element $(\overline{u}, \overline{p^*}) \in U\times Y^*$ is called saddle point of $L$ if 
	\[
		L(\overline{u}, p^*) \leq L(\overline{u}, \overline{p^*}) \leq L(u, p^*), \quad \forall u \in U, \forall p^* \in Y^*.
	\]
\end{definition}

\begin{theorem}
	Assume that $\Phi$ convex and weak lower semicontinuous. Then $(u^*, \overline{p^*})$ is a saddle point of $L$ if and only if $\overline{u}$ is solution of \eqref{eq8. P}, $\overline{p^*}$ is solution of \eqref{eq8. P*} and $\inf\eqref{eq8. P}=\sup\eqref{eq8. P*}$.
	\begin{proof}
		Let $(\overline{u}, \overline{p^*})$ be a saddle point of $L$. We have that, 
		\begin{align*}
		\left.
		\begin{array}{c}
		L(\overline{u}, \overline{p^*}) = \inf_{u \in U} L(u, \overline{p^*})=-\Phi^*(0, \overline{p^*}) \\
		L(\overline{u}, \overline{p^*}) = \sup_{p^* \in Y^*} L(\overline{u}, \overline{p^*})=-\Phi^*(\overline{u}, 0)
		\end{array}
		\right\rbrace \implies \Phi(\overline{u}, 0)+\Phi^*(0, \overline{p}^*)=0
		\end{align*}
		Theorem about extremal conditions $\implies$ $\overline{u}$ is a solution of $\eqref{eq8. P}$, $\overline{p^*}$ solution of $\eqref{eq8. P*}$ and 
		\[
			\inf\eqref{eq8. P} = \sup\eqref{eq8. P*}
		\]
		"other direction" follows the same argumentation.
	\end{proof}
\end{theorem}
\begin{theorem}[Saddle point theorem.]
	Let $\Phi: U \times Y \rightarrow \overline{\mathbb{R}}$ be convex, weak lower semicontinuous and $\eqref{eq8. P}$ is stable. Then $\overline{u}\in U$ is a solution of $\eqref{eq8. P}$ if and only if then exist $\overline{p^*} \in Y^*$ such that $(\overline{u}, \overline{p^*})$, is a saddle point of $L$.
	\begin{proof}
		Out of the scope of the course. According to professor.
	\end{proof}
\end{theorem}

\section{Lecture 10}
\subsection{Minimization}
Two approaches for the solution of infinite-dimensional optimization problems.

\begin{itemize}
	\item \textbf{Discretize then Optimize:} this approach consists of a direct discretization of the problem which leads to a non-linear programming problem in a finite subspace $U_n \subset U$. If the discretization is accurate enough, the approximate solution $\overline{u_n} \in U_n$ is close to the real solution $\overline{u} \in U$
	
	\item \textbf{Optimize then Discretize:} The idea is to formulate the problem in infinite dimensions and to apply discretization only for the solution of sub-problems and for the evluation of the objective function. The main advantage is that quantitative estimates for the convergence of the optimization method can be combined with error estimates of the discretization.
\end{itemize}
The general framework for minimization algorithms for 
\[
	\min_{u\in U} J(u)
\]
with $J:U\rightarrow \overline{\mathbb{R}}$, $U$ Banach space delivers a sequence $(u_k)_k \in U$.

\paragraph{Desired Concepts}
\begin{enumerate}
	\item \textbf{Global convergence: }We need to measure if a limit is a candidate for a solution of \eqref{eq8. P}.
	\begin{definition}[Stationary measure]
		Let $\Sigma: U\rightarrow \mathbb{R}_+$ a functional, given by
		\begin{align*}
		\Sigma(u)=
		\left\lbrace	
		\begin{array}{ll}
			0 & \quad u \text{ is stationary point}.\\
			\alpha(u) \in \mathbb{R}_+ & \quad \text{everywhere else}			
			\end{array}
		\right.
		\end{align*}
	We call this function an stationary measure. 
	\end{definition}
	\begin{example}\
	\begin{itemize}
		\item For unconstrained problems: \[\Sigma(u)=\norm{\nabla J(u)}_{U^*}\]
		\item In the case $U$ is a Hilbert space and the problem restricted over a closed and convex set $C\subset U$. 
		\[
		\Sigma(u)= \norm{u-P_c(u-\nabla J(u))}
		\]
	\end{itemize}
	\end{example}
	We have global convergence if:
	\begin{itemize}
		\item Every accumulation point of $(u_k)_k$ is a stationary point.
		\item $\lim_{k\rightarrow \infty} \Sigma(u_k)=0$ for some continuous stationary measure.
		\item $(u_k)_k$ has an accumulation point which is stationary.
		\item $\liminf_{k\rightarrow\infty} \Sigma(u_k) =0$ and $\Sigma$ continuous.
	\end{itemize}
	\item \textbf{Fast local convergence:}
	Let $\overline{u}$ be a stationary point of $\eqref{eq8. P}$. We say that $u_k \rightarrow \overline{u}$ locally with \textit{q-superlinear rate} if $\norm{u_{k+1}-\overline{u}}_U=C_k\norm{u_k-\overline{u}}_U$, as $k\rightarrow \infty$ in a neighborhood of $\overline{u}$ and $C_k\rightarrow 0$ as $k\rightarrow 0$.
	We have convergence of order $\alpha +1$, $\alpha >0$ if 
	$\norm{u_{k+1}-\overline{u}}_U=C_k\norm{u_k-\overline{u}}^{\alpha+1}_U$. With $\alpha =1$ we obtain \textit{q-quadratic} convergence.
\end{enumerate}


\subsection{Gradient methods}
In the following, we assume $H$ to be Hilbert space.
\paragraph{Motivation}
	Dynamical systems in physics are often based on the idea of gradient flow with respect to the energy which the system follows. Consider for example the heat equation. Hence, the thermal energy is given by
	\[
		E(u)=\frac{1}{2}\int \abs{\nabla u}^2 dx.
	\]
	
	The gradient flow is defined by,
	\[
		\frac{\partial u}{\partial t} = -E(u),
	\]
	which yields the heat equation.
	\[
		\dot{u}(t)=\Delta u
	\]
In order to obtain a minimization method in $U$, we introduce the gradient flow $\frac{\partial u}{\partial t} = -\nabla J(u)$, with $\nabla J(u)\in H$ which can be associated with the gradient of $J$ at $u$, i.e.

\[
	\left\langle  v,\pdev{u}{t} \right\rangle = \left\langle v, \nabla J(u)\right\rangle = -J'(u;v), \quad \forall v \in H
\]

The evolution of $J$ corresponding to the gradient flow is given by

\[
	\frac{\partial}{\partial t} \left(J(u)\right) = \left\langle \pdev{u}{t}, \nabla J(u) \right\rangle = -\norm{\pdev{u}{t}}^2 \leq 0
\]

i.e. the objective function is decreasing and \[\pdev{}{t} J(u) =0 \iff \pdev{u}{t}=0\].

Consequently, the gradient flow will decrease the objective function $J$ until the evolution arrives a stationary point. In order to obtain an iterative optimization method, we use an explicit time discretization of the flow, i.e.
\begin{align*}
u_{k+1}=u_k - \tau_k\nabla J(u_k) &\quad \text{for } k=0,1,2,.. \\
\nabla J(u_{k+1})=0 &\quad \text{Stop the iteration.}
\end{align*}
with appropriate (small) choice of the time step $\tau_k >0$. 
\begin{remark}
We are only interested in the minimization of the objective function and not in the accurate approximation of the solution of the gradient flow.
\end{remark}


Therefore we select the step size purely based on the suitable descent of objective function. A classical way to do this is the so called \textit{Armijo-Goldstein} rules, which are based on the ``effective descent",
\[
\mathcal{D}_{\text{eff}}(\tau) = J(u_k+\tau s)-J(u_k)
\]
and the ``expected descent"
\[
	\mathcal{D}_{\text{exp}} = \tau J'(u_k;s)=\tau \langle s, \nabla J(u_k)\rangle
\]
where  $s=-\nabla J(u_k)$. They are related to each other by the Taylor formula,
\[
	\mathcal{D}_{\text{eff}}(\tau)=\mathcal{D}_{\text{exp}}(\tau)+O(\tau)
\]
Therefore, we can test if 
\begin{equation}
\alpha \mathcal{D}_{\text{exp}}(\tau)\leq \mathcal{D}_{\text{eff}}(\tau) \leq \beta \mathcal{D}_{\text{exp}}(\tau) \label{eq10. DexpDeffDexp} \tag{*}
\end{equation}
with constants $0<\beta<\alpha<1$. 

If $\mathcal{D}_{\text{eff}}(\tau) > \beta \mathcal{D}_{\text{exp}}(\tau)$, then $\tau$ is too large and we decrease it. 
If $\mathcal{D}_{\text{eff}}(\tau)<\alpha\mathcal{D}_{\text{exp}}(\tau)$, we could still increase $\tau$. We accept $\tau$, if 
\eqref{eq10. DexpDeffDexp} is fulfilled. Typical choices of the constants are $\alpha \approx 0.9$ and $\beta \approx 0.1$.

Moreover, the strategy for increasing and decreasing $\tau$ should be different, i.e. multiplying with $1.5$, for increasing and dividing by $2$ for decreasing.

\begin{theorem}
	Let $J:H\rightarrow \overline{\mathbb{R}}$ be twice Fr\'echet-differentiable and weakly lower semicontinuous on a Hilbert space $H$. Moreover, let the level sets,
	\[
		\mu_\xi = \{ u \in U \mid J(u)\leq \xi \}
	\]
	be bounded in $H$ for each $\xi \in \mathbb{R}$ and empty for sufficiently small $\xi$. Then the sequence $(u_k)_k$ generated by the gradient method with the \textit{Amijo-Goldstein} line search has a weakly convergent subsequence, whose limit is a stationary point. 
	
	\begin{proof}
		Since the gradient method is a descent method we have $J(u_k) \leq J(u_0)$, $\forall k \geq 0$,
		i.e. $(u_k)_k$ is bounded, the lemma \eqref{lemma0. Bounded and weakly convergent}, implies that $\exists (u_{k_l})_l \xrightharpoonup[l\rightarrow\infty]{} \overline{u}$.
		
		Therefore,
		\begin{align*}
			\sum_{k=0}^N \norm{u_{k+1}-u_k}^2 &= \sum_{k=0}^N \langle u_{k+1}-u_{k}, -\tau_k \nabla J(u_k)\rangle
		\end{align*}
		Considering that,
		\begin{align*}
			\frac{1}{\beta} \mathcal{D}_{\text{eff}}(\tau)=\frac{1}{\beta}\left(J(u_k+\tau s)-J(u_k)\right) \leq \mathcal{D}_{\text{exp}}(\tau)=\tau J'(u_k;s)
		\end{align*}
		Then we have substituting, 
		\begin{align*}
						\sum_{k=0}^N \norm{u_{k+1}-u_k}^2&\leq \frac{1}{\beta} \sum_{k=0}^{N} (J(u_k)-J(u_{k+1})) \\&= \frac{1}{\beta} (J(u_0)-J(u_{N+1})) \\
			&\leq \frac{1}{\beta} (J(u_0)-\inf_{u\in U} J(u)) = q
		\end{align*}
	$q$ is independent of $N$; for $N\rightarrow \infty$ we obtain 
	\[
	\sum_{l=0}^{\infty} \norm{u_{{k_l}+1}-u_{k_l}}^2=\sum_{k=0}^{\infty} \norm{u_{k+1}-u_{k}}^2\leq q
	\]
	
	Thus, there is a subsequence $(u_k)_l$, such that
	\[
	\norm{\tau_{k_l}\nabla J(u_{k_l})}=\norm{u_{{k_l}+1}-u_{k_l}} \xrightarrow{l\rightarrow \infty} 0.
	\]
	
	$\exists c >0$, $J''(u_{k_l};(v,v))\leq c \norm{v}^2$, $\forall v \in U$. 	Amijo-Goldstein implies,
	\begin{align*}
		\alpha \mathcal{D}_{\text{exp}} (\tau) &= \alpha J'(u_{k_l};u_{{k_l}+1}-u_{{k_l}})\\ &\leq \alpha \mathcal{D}_{\text{exp}}(\tau_{k_l})\\&= J(u_{k_l+1})-J(u_{k_l}) \\
		&= J'(u_{k_l};u_{k_l+1}-u_{k_l}) + \int_{0}^{1} J''\left(u_{k_l}+t(u_{k_l+1}-u_{k_l}); \left((u_{k_l+1}-u_{k_l}, u_{k_l+1}-u_{k_l})\right)\right) dt \\
		& \leq J'(u_{k_l};u_{k_l+1}-u_{k_l}) + C \norm{u_{k_l+1}-u_{k_l}}^2
		\end{align*}
	implies,
	\begin{align*}
	(\alpha -1) J'(u_{k_l};u_{k_l+1}-u_{k_l}) &\leq C \norm{u_{k_l+1}-u_{k_l}}^2 
	\end{align*}
	Since $u_{k_l+1}-u_{k_l}= -\tau_k \nabla J(u_{k_l})$.
	\begin{align*}
		(1-\alpha) \tau_{k_l} \norm{\nabla J(u_{k_l})}^2 \leq c \tau_{k_l}^2\norm{\nabla J(u_{k_l})}^2
	\end{align*}
	
	Therefore, $\nabla J(u_{k_l})=0$, or $ 1-\alpha \leq c \tau_{k_l}$. But since $1-\alpha > 0$ and $c <0$. 
	We have $\nabla J(u_{k_l}) =0$, implying the algorithm reached stationary point; 
		\[\forall j \geq k_l \quad u_j= u_{k_l}  \implies \norm{\nabla J(u_{k_l})} \rightarrow 0 \implies \nabla J(\overline{u})=0\].
	\end{proof}
\end{theorem}

\section*{Symbols}
\begin{tabular}{ll}
	$\mathbb{N}$ & Natural numbers \\
	$\mathbb{R}$ & The real line $(-\infty, \infty)$ \\
	$\mathbb{R}_+$ & Positive real numbers $(0,\infty)$ \\
	$\mathbb{R}_-$ & Negative real numbers $(-\infty,0)$ \\
	$\overline{\mathbb{R}}$ & Extended real line $[-\infty, \infty]$ \\
	$\mathsf{D} J(u;v)$ & First Derivative at $u$ with direction $v$. \\
	$\dom J$	& Effective Domain of $J$. \\
	$(x_k)_k$   & Sequence with $k \in \mathbb{N}$ indexing the elements $x_k$. \\
	$(x_{k_l})_l$ & Subsequence of a sequence $(x_k)_k$, indexed by $l \in \mathbb{N}$ \\
	$\mathcal{L}(X,Y)$ & Set of linear operators from $X$ to $Y$. \\
	$\mathcal{B}(X,Y)$ & Set of linear and bounded operators from $X$ to $Y$. \\
	$C^k(X,Y)$ & Set of continuously $k$-differentiable operators from $X$ to $Y$. If $Y$ is not specified assume the real line. \\
	$C^k_0(X,Y)$ & Set of continuously $k$-differentiable operators from $X$ to $Y$ with compact support. \\
	$\epi J$ & Epigraph of $J$. \\
	$\text{gra}\, J$ & Graph of $J$.\\
	$I_C$ & Indicator function of the set $C$. \\
	$\overline{C}$ & Closure of  the set $C$ \\
	$A^*$ & Adjoint of operator $A$.\\
	$f^*$ & Convex conjugate of the functional $f$.
\end{tabular}
\begin{thebibliography}{5}
\bibitem{Brandt} Lecture notes of Prof. Dr. Anna Christina Brandt.
\bibitem{Barbu} V. Barbu and Th. Precupanu, Convexity and optimization in Banach spaces.
\bibitem{Ekeland} I. Ekeland and R. Teman, Convex analysis and variational problems.
\bibitem{Bauschke} H. Bauschke and P. Combettes, Convex analysis and monotone operator theory in Hilbert spaces.
\end{thebibliography}

\end{document}
